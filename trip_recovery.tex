\chapter{Trip Recovery} \label{ch:trip_recovery}

\section{Amputee Trip Recovery} \label{sec:amputee_trip_recovery}

\begin{enumerate} \item \papertitle{Transfemoral amputee recovery strategies
following trips to their sound and prosthesis sides throughout swing
phase}{shirota2015transfemoral}{-6ex}. 
    
    \takeaway{Stance side behavior possibly more important than swing behavior
    for trip recovery.} 

    \takeaway{Amputees can still implement all able-bodied
    strategies including elevating strategies}
    
    In this paper the researchers tripped amputees on both their healthy and
    prosthesis sides and tripped healthy subjects on both sides using a
    cable-driven tripping machine. They identified four strategies for able
    bodied subjects: 
    
    \begin{description} 
        \item[Elevating strategy] Subject raised foot and placed it ahead of
        arrested location.  

        \item[Delayed-lowering strategy] Subject raised foot and put it behind
        arresting location 

        \item[Lowering strategy] Subject did not raise foot and placed it at or
        behind arresting location 

        \item[Incomplete Arrest] Subject did not raise foot and placed it ahead
        of arrested location
    \end{description}
    
    Additionally the authors identified two amputee-only strategies:
    \begin{description} 
        \item[Hopping strategy] Subject jumped over virtual obstacle with both
        feet.  

        \item[Skipping strategy] Subject lowered tripped foot and then took an
        extra step with the tripped foot over the obstacle.
    \end{description}
    
    Most individual amputees used a subset of the able-bodied strategies. But
    across all subjects, all able-bodied strategies were utilized. Surprisingly,
    foot kinematics were not different between amputees and healthy-subjects.
    \ie amputees were able to implement all strategies even on the prosthesis
    via sound-side leg and hip movements.
    
    However, the distribution of strategies is different especially when the
    sound side is tripped (amputees used elevating and delayed lowering
    strategies less and used them even less on the sound side). The author's
    credit the large change in the sound side to the lack of active torque on
    the prosthesis side when it is the stance leg. The prosthesis cannot produce
    counteracting moments to halt forward momentum.  
    
\end{enumerate}

\section{Trip Classification} 

Trip Classification papers successfully detect trips and classify them based on
strategy. So far these studies gather data from able-bodied subjects. These
results may not be applicable to amputee subjects.
However,~\citet{shirota2015transfemoral} note the similarity in kinematics of
trip recoveries between amputees and able-bodied subjects so they still might
work.  The main issue is likely the DAGGER problem: data was collected from
subjects who are correctly performing trip recovery. The distribution of states
on a prosthesis who's trip recovery controller makes mistakes will be different
than that of an able-bodied subject. Therefore, the IID assumption is violated
and this should not be treated as a pure supervised learning problem.

\begin{enumerate} 
    \item \papertitle{Stumble detection and classification for an intelligent
    transfemoral prosthesis}{lawson2010stumble}{-3ex}. 

    \takeaway{Successfully implemented trip detection with average delay of 70 ms.}
   
    Collected data from healthy subjects and trained trip detection and
    classification into lowering and elevating strategies. Used FFT features
    in windows of hip, knee, and ankle data of tripped leg. The author's trained
    and tested on the same data so the results are suspect.

    \item \papertitle{Recovery strategy identification throughout swing phase
    using kinematic data from the tripped leg}{shirota2014recovery}{-6ex}. 

    \takeaway{Two part classifier that first classifies walking from trip and
    then the type of trip works better than one part classifier.}
    
    Researcher tripped able-bodied subjects and learned one and two part
    classifiers to classify trip and type of trip. Two part classifier worked
    better. Classifier was learned for each specific subject. Features used were
    min, max, mean, and standard deviation of knee, ankle, foot angle/position
    and acceleration. Features were calculated within windows. Optimal number of
    windows and window length were identified.

\end{enumerate}
%%

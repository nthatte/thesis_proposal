\chapter{Prosthesis Design}\label{ch:Prosthesis_Design}

\section{Transtibial Prostheses}\label{sec:Transtibial_Prosthesis_Design}
\begin{enumerate} 
    \item \papertitle{Biomechanical design of a powered ankle-foot
    prosthesis}{au2007biomechanical}{-3ex}

    \takeaway{It is possible to design a powered ankle prosthesis that can
    deliver the required instantaneous power by using an SEA with a parallel
    spring.}

    A powered ankle prosthesis is designed to be roughly the same weight as a
    nominal human ankle-foot, reproduce the torques/speeds seen during normal
    human walking, and be tolerant to the shock at heel-strike. They specify
    that the torque output should be capable of outputting torques between 50 Nm
    and 140 Nm at 3.5 Hz. Adding a series spring helps improve the torque
    bandwidth because the SEA doesn't have to have as high torques to achieve
    the same ankle torque output. (Large Force Bandwidth).

    \item \papertitle{Powered ankle-foot prosthesis}{au2008powered}{0ex}

    \takeaway{Series + parallel compliance allows the prosthesis to have shock
    tolerance and still meet bandwidth requirements.}

    A powered ankle prosthesis is designed to be roughly the same weight as a
    nominal human ankle-foot, reproduce the torques/speeds seen during normal
    human walking, and be tolerant to the shock at heel-strike. They specify
    that the torque output should be capable of outputting torques between 50 Nm
    and 140 Nm at 3.5 Hz. To ensure that the prostheses dynamics do not
    interfere they further increase the bandwidth requirement to 17.5 Hz. They
    find a series that is soft enough to ensure they don't exceed the shock
    rating of the gear box would not meet the torque bandwidth requirements.
    Adding a series spring helps improve the torque bandwidth because the SEA
    doesn't have to have as high torques to achieve the same ankle torque
    output.

    \item \papertitle{Powered Lower Limb Prostheses}{grimmer2015powered}{0ex}

    \takeaway{Series and parallel compliance architectures allows prostheses to
    reproduce both walking and running peak power with much less motor peak
    power.}

    For running and even walking, peak power requirements are quite high.
    Springs allow energy to be stored in one part of gait and released in
    another, and thus energy production can be distributed over more time,
    reducing peak power output. This allows for the use of smaller, lighter
    motors, resulting in more practical designs. However, this sort of design is
    less versatile as it is harder to control timing of power production; it is
    controlled implicitly by the spring dynamics.

    \item \papertitle{SPARKy 3: Design of an active robotic ankle prosthesis
    with two actuated degrees of freedom using regenerative
    kinetics}{bellman2008sparky}{-6ex}

    \takeaway{Using two motors allows for inversion/eversion control and
    high peak power output. Series springs further increase the potential peak
    output}

    \item \papertitle{A walking controller for a powered ankle
    prosthesis}{shultz2014walking}{-3ex}

    \takeaway{Ankle design features a parallel spring}

    With impedance control and they reproduce healthy walking gait kinematics
    and kinetics.
\end{enumerate}

\section{Transfemoral Prostheses}\label{sec:Transfemoral_Prosthesis_Design}
\begin{enumerate} 
    \item \papertitle{Design and control of a powered knee and ankle
    prosthesis}{sup2007design}{-3ex}

    \takeaway{First powered knee and ankle prosthesis} 

    This was the first powered knee-ankle prosthesis by Goldfarb's group. The
    device uses a pneumatic mono-propellant based actuator and impedance
    control. It was very light (<3 kg) but had off-board compression of the gas.
    It had torque control via axial load cells in the pistons. The resultant
    torque/power curves don't look that natural

    \item \papertitle{Preliminary evaluations of a self-contained
    anthropomorphic transfemoral prosthesis}{sup2009preliminary}{-6ex}

    \takeaway{This is the second iteration of the Vanderbilt prosthesis and is
    the first to be stand-alone} 

    This was the second powered knee-ankle prosthesis by Goldfarb's group. The
    device uses a Maxon EC30 and ball screw transmissions. It has load cells in
    series with the actuators. Additionally, it features an ankle parallel
    spring.
    
\end{enumerate}

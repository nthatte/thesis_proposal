%%%%%%%%%%%%%%%%%%%%%%%%%%%%%%%%%%%%%%%%%%%%%%%%%%%%%%%%%%%%%%%%%%%%%%
% How to use writeLaTeX: 
%
% You edit the source code here on the left, and the preview on the
% right shows you the result within a few seconds.
%
% Bookmark this page and share the URL with your co-authors. They can
% edit at the same time!
%
% You can upload figures, bibliographies, custom classes and
% styles using the files menu.
%
% If you're new to LaTeX, the wikibook is a great place to start:
% http://en.wikibooks.org/wiki/LaTeX
%
%%%%%%%%%%%%%%%%%%%%%%%%%%%%%%%%%%%%%%%%%%%%%%%%%%%%%%%%%%%%%%%%%%%%%%
\documentclass{tufte-book}

%\hypersetup{colorlinks}% uncomment this line if you prefer colored hyperlinks (e.g., for onscreen viewing)

%%
% Book metadata
\title{List of Papers}
\author[Nitish Thatte]{Nitish Thatte}
%\publisher{Publisher of This Book}

%%
% If they're installed, use Bergamo and Chantilly from www.fontsite.com.
% They're clones of Bembo and Gill Sans, respectively.
%\IfFileExists{bergamo.sty}{\usepackage[osf]{bergamo}}{}% Bembo
%\IfFileExists{chantill.sty}{\usepackage{chantill}}{}% Gill Sans

\usepackage{microtype}

%%
% For nicely typeset tabular material
\usepackage{booktabs}

%%
% For graphics / images
\usepackage{graphicx}
\setkeys{Gin}{width=\linewidth,totalheight=\textheight,keepaspectratio}
\graphicspath{{graphics/}}

% The fancyvrb package lets us customize the formatting of verbatim
% environments.  We use a slightly smaller font.
\usepackage{fancyvrb}
\fvset{fontsize=\normalsize}

\usepackage{amsmath} % assumes amsmath package installed
\usepackage{amssymb}  % assumes amsmath package installed
%%
% Prints argument within hanging parentheses (i.e., parentheses that take
% up no horizontal space).  Useful in tabular environments.
\newcommand{\hangp}[1]{\makebox[0pt][r]{(}#1\makebox[0pt][l]{)}}

%%
% Prints an asterisk that takes up no horizontal space.
% Useful in tabular environments.
\newcommand{\hangstar}{\makebox[0pt][l]{*}}

%%
% Prints a trailing space in a smart way.
\usepackage{xspace}

%%
% Some shortcuts for Tufte's book titles.  The lowercase commands will
% produce the initials of the book title in italics.  The all-caps commands
% will print out the full title of the book in italics.

% Prints the month name (e.g., January) and the year (e.g., 2008)
\newcommand{\monthyear}{%
  \ifcase\month\or January\or February\or March\or April\or May\or June\or
  July\or August\or September\or October\or November\or
  December\fi\space\number\year
}


% Prints an epigraph and speaker in sans serif, all-caps type.
\newcommand{\openepigraph}[2]{%
  %\sffamily\fontsize{14}{16}\selectfont
  \begin{fullwidth}
  \sffamily\large
  \begin{doublespace}
  \noindent\allcaps{#1}\\% epigraph
  \noindent\allcaps{#2}% author
  \end{doublespace}
  \end{fullwidth}
}

% Inserts a blank page
\newcommand{\blankpage}{\newpage\hbox{}\thispagestyle{empty}\newpage}

\usepackage{units}

% Typesets the font size, leading, and measure in the form of 10/12x26 pc.
\newcommand{\measure}[3]{#1/#2$\times$\unit[#3]{pc}}

% Macros for typesetting the documentation
\newcommand{\hlred}[1]{\textcolor{Maroon}{#1}}% prints in red
\newcommand{\hangleft}[1]{\makebox[0pt][r]{#1}}
\newcommand{\hairsp}{\hspace{1pt}}% hair space
\newcommand{\hquad}{\hskip0.5em\relax}% half quad space
\newcommand{\TODO}{\textcolor{red}{\bf TODO!}\xspace}
\newcommand{\ie}{\textit{i.\hairsp{}e.}\xspace}
\newcommand{\eg}{\textit{e.\hairsp{}g.}\xspace}
\newcommand{\na}{\quad--}% used in tables for N/A cells
\providecommand{\XeLaTeX}{X\lower.5ex\hbox{\kern-0.15em\reflectbox{E}}\kern-0.1em\LaTeX}
\newcommand{\tXeLaTeX}{\XeLaTeX\index{XeLaTeX@\protect\XeLaTeX}}
% \index{\texttt{\textbackslash xyz}@\hangleft{\texttt{\textbackslash}}\texttt{xyz}}
\newcommand{\tuftebs}{\symbol{'134}}% a backslash in tt type in OT1/T1
\newcommand{\doccmdnoindex}[2][]{\texttt{\tuftebs#2}}% command name -- adds backslash automatically (and doesn't add cmd to the index)
\newcommand{\doccmddef}[2][]{%
  \hlred{\texttt{\tuftebs#2}}\label{cmd:#2}%
  \ifthenelse{\isempty{#1}}%
    {% add the command to the index
      \index{#2 command@\protect\hangleft{\texttt{\tuftebs}}\texttt{#2}}% command name
    }%
    {% add the command and package to the index
      \index{#2 command@\protect\hangleft{\texttt{\tuftebs}}\texttt{#2} (\texttt{#1} package)}% command name
      \index{#1 package@\texttt{#1} package}\index{packages!#1@\texttt{#1}}% package name
    }%
}% command name -- adds backslash automatically
\newcommand{\doccmd}[2][]{%
  \texttt{\tuftebs#2}%
  \ifthenelse{\isempty{#1}}%
    {% add the command to the index
      \index{#2 command@\protect\hangleft{\texttt{\tuftebs}}\texttt{#2}}% command name
    }%
    {% add the command and package to the index
      \index{#2 command@\protect\hangleft{\texttt{\tuftebs}}\texttt{#2} (\texttt{#1} package)}% command name
      \index{#1 package@\texttt{#1} package}\index{packages!#1@\texttt{#1}}% package name
    }%
}% command name -- adds backslash automatically
\newcommand{\docopt}[1]{\ensuremath{\langle}\textrm{\textit{#1}}\ensuremath{\rangle}}% optional command argument
\newcommand{\docarg}[1]{\textrm{\textit{#1}}}% (required) command argument
\newenvironment{docspec}{\begin{quotation}\ttfamily\parskip0pt\parindent0pt\ignorespaces}{\end{quotation}}% command specification environment
\newcommand{\docenv}[1]{\texttt{#1}\index{#1 environment@\texttt{#1} environment}\index{environments!#1@\texttt{#1}}}% environment name
\newcommand{\docenvdef}[1]{\hlred{\texttt{#1}}\label{env:#1}\index{#1 environment@\texttt{#1} environment}\index{environments!#1@\texttt{#1}}}% environment name
\newcommand{\docpkg}[1]{\texttt{#1}\index{#1 package@\texttt{#1} package}\index{packages!#1@\texttt{#1}}}% package name
\newcommand{\doccls}[1]{\texttt{#1}}% document class name
\newcommand{\docclsopt}[1]{\texttt{#1}\index{#1 class option@\texttt{#1} class option}\index{class options!#1@\texttt{#1}}}% document class option name
\newcommand{\docclsoptdef}[1]{\hlred{\texttt{#1}}\label{clsopt:#1}\index{#1 class option@\texttt{#1} class option}\index{class options!#1@\texttt{#1}}}% document class option name defined
\newcommand{\docmsg}[2]{\bigskip\begin{fullwidth}\noindent\ttfamily#1\end{fullwidth}\medskip\par\noindent#2}
\newcommand{\docfilehook}[2]{\texttt{#1}\index{file hooks!#2}\index{#1@\texttt{#1}}}
\newcommand{\doccounter}[1]{\texttt{#1}\index{#1 counter@\texttt{#1} counter}}

%my commands
\synctex=1
\newcommand{\papertitle}[3]{\citet{#2}, \textsc{#1} \cite[#3]{#2}} %paper title formatting
\newcommand{\takeaway}[1]{\textbf{#1}} %paper title formatting
\newcommand{\prob}[1]{\mathrm{P} \left( #1 \right)}
\newcommand{\func}[2]{\mathrm{#1}\left( #2 \right)}

% Generates the index
\usepackage{makeidx}
\makeindex

\begin{document}

% Front matter
\frontmatter

% r.3 full title page
\maketitle

% v.4 copyright page
\newpage
\begin{fullwidth}
~\vfill
\thispagestyle{empty}
\setlength{\parindent}{0pt}
\setlength{\parskip}{\baselineskip}
Copyright \copyright\ \the\year\ \thanklessauthor

\par\smallcaps{Published by \thanklesspublisher}

%\par\smallcaps{tufte-latex.googlecode.com}

\par Licensed under the Apache License, Version 2.0 (the ``License''); you may not
use this file except in compliance with the License. You may obtain a copy
of the License at \url{http://www.apache.org/licenses/LICENSE-2.0}. Unless
required by applicable law or agreed to in writing, software distributed
under the License is distributed on an \smallcaps{``AS IS'' BASIS, WITHOUT
WARRANTIES OR CONDITIONS OF ANY KIND}, either express or implied. See the
License for the specific language governing permissions and limitations
under the License.\index{license}

\par\textit{First printing, \monthyear}
\end{fullwidth}

% r.5 contents
\tableofcontents

\listoffigures

\listoftables

% r.7 dedication
%\cleardoublepage
%~\vfill
%\begin{doublespace}
%\noindent\fontsize{18}{22}\selectfont\itshape
%\nohyphenation
%Dedicated to those who appreciate \LaTeX{} 
%and the work of \mbox{Edward R.~Tufte} 
%and \mbox{Donald E.~Knuth}.
%\end{doublespace}
%\vfill
%\vfill

% r.9 introduction
\cleardoublepage
\chapter*{Introduction}

This document contains descriptions of relevant work.


%%
% Start the main matter (normal chapters)
\mainmatter
% The back matter contains appendices, bibliographies, indices, glossaries, etc.
\chapter{Prosthesis Design}\label{ch:Prosthesis_Design}

\section{Transtibial Prostheses}\label{sec:Transtibial_Prosthesis_Design}
\begin{enumerate} 
    \item \papertitle{Biomechanical design of a powered ankle-foot
    prosthesis}{au2007biomechanical}{-3ex}

    \takeaway{It is possible to design a powered ankle prosthesis that can
    deliver the required instantaneous power by using an SEA with a parallel
    spring.}

    A powered ankle prosthesis is designed to be roughly the same weight as a
    nominal human ankle-foot, reproduce the torques/speeds seen during normal
    human walking, and be tolerant to the shock at heel-strike. They specify
    that the torque output should be capable of outputting torques between 50 Nm
    and 140 Nm at 3.5 Hz. Adding a series spring helps improve the torque
    bandwidth because the SEA doesn't have to have as high torques to achieve
    the same ankle torque output. (Large Force Bandwidth).

    \item \papertitle{Powered ankle-foot prosthesis}{au2008powered}{0ex}

    \takeaway{Series + parallel compliance allows the prosthesis to have shock
    tolerance and still meet bandwidth requirements.}

    A powered ankle prosthesis is designed to be roughly the same weight as a
    nominal human ankle-foot, reproduce the torques/speeds seen during normal
    human walking, and be tolerant to the shock at heel-strike. They specify
    that the torque output should be capable of outputting torques between 50 Nm
    and 140 Nm at 3.5 Hz. To ensure that the prostheses dynamics do not
    interfere they further increase the bandwidth requirement to 17.5 Hz. They
    find a series that is soft enough to ensure they don't exceed the shock
    rating of the gear box would not meet the torque bandwidth requirements.
    Adding a series spring helps improve the torque bandwidth because the SEA
    doesn't have to have as high torques to achieve the same ankle torque
    output.

    \item \papertitle{Powered Lower Limb Prostheses}{grimmer2015powered}{0ex}

    \takeaway{Series and parallel compliance architectures allows prostheses to
    reproduce both walking and running peak power with much less motor peak
    power.}

    For running and even walking, peak power requirements are quite high.
    Springs allow energy to be stored in one part of gait and released in
    another, and thus energy production can be distributed over more time,
    reducing peak power output. This allows for the use of smaller, lighter
    motors, resulting in more practical designs. However, this sort of design is
    less versatile as it is harder to control timing of power production; it is
    controlled implicitly by the spring dynamics.

    \item \papertitle{SPARKy 3: Design of an active robotic ankle prosthesis
    with two actuated degrees of freedom using regenerative
    kinetics}{bellman2008sparky}{-6ex}

    \takeaway{Using two motors allows for inversion/eversion control and
    high peak power output. Series springs further increase the potential peak
    output}

    \item \papertitle{A walking controller for a powered ankle
    prosthesis}{shultz2014walking}{-3ex}

    \takeaway{Ankle design features a parallel spring}

    With impedance control and they reproduce healthy walking gait kinematics
    and kinetics.
\end{enumerate}

\section{Transfemoral Prostheses}\label{sec:Transfemoral_Prosthesis_Design}
\begin{enumerate} 
    \item \papertitle{Design and control of a powered knee and ankle
    prosthesis}{sup2007design}{-3ex}

    \takeaway{First powered knee and ankle prosthesis} 

    This was the first powered knee-ankle prosthesis by Goldfarb's group. The
    device uses a pneumatic mono-propellant based actuator and impedance
    control. It was very light (<3 kg) but had off-board compression of the gas.
    It had torque control via axial load cells in the pistons. The resultant
    torque/power curves don't look that natural

    \item \papertitle{Preliminary evaluations of a self-contained
    anthropomorphic transfemoral prosthesis}{sup2009preliminary}{-6ex}

    \takeaway{This is the second iteration of the Vanderbilt prosthesis and is
    the first to be stand-alone} 

    This was the second powered knee-ankle prosthesis by Goldfarb's group. The
    device uses a Maxon EC30 and ball screw transmissions. It has load cells in
    series with the actuators. Additionally, it features an ankle parallel
    spring.
    
\end{enumerate}

\chapter{Prosthesis Control} \label{ch:pros_control}

\section{EMG Control} \label{sec:emg_pros_control}

\begin{enumerate} 
    \item \papertitle{A Strategy for Identifying Locomotion Modes Using Surface
    Electromyography}{huang2009strategy}{-3ex}. 

    \takeaway{Paper suggests using Linear Discriminant Analysis instead of
    Artificial Neural Networks, combining EMG time domain or autoregressive
    features with kinematics data as in the later work
    \citet{hargrove2015intuitive}}
    
    In this work EMG sensors were used to measure neuromuscular data from 8
    healthy subjects and 2 TF amputees performing different walking modes. The
    authors used Linear Discriminiant Analysis (LDA) and Artificial Neural
    Networks (ANN) trained on time domain (TD), Autoregressive (AR) features,
    and a combination of both (TDAR) to recognize walking mode during 4
    sub-phases of gait. Different subsets of the EMG sensors were tried and
    different. They found that LDA had lower classification error than ANN and
    was simpler. They found using hip muscles in addition to thigh muscles
    helped even though these would probably be more difficult to measure outside
    of lab. Also they found, using only TD or only AR features worked better
    than using both. A window size of 120ms or more is less noisy and small
    increments between window starts allows for faster updates.  Although, later
    work in~\citet{hargrove2015intuitive} uses a weighted average of
    instantaneous features instead of discrete windows. Also the later work adds
    in mechanical data as is suggested by this work to improve performance. This
    work only reports LOOCV offline validation. The later work also includes
    online validation with slightly worse performance. Dagger may help.  

    \item \papertitle{Multiclass real-time intent recognition of a powered lower
    limb prosthesis}{varol2010multiclass}{-3ex}. 

    \takeaway{Using mechanical sensor data (no emg), LDA instead of PCA for
    dimensionality reduction and Gaussian mixture model the authors built a gait
    intent recognition system}

    The authors built a gait recognition system based on mechanical sensor data.
    They used LDA for dimensionality which is similar to logistic regression. ie
    the normal vector for a dividing hyper plane will tell you which dimensions
    are important and which are not. They used a Gaussian mixture model trained
    with EM to fit the classes. They only trained on tested with one subject and
    report error rates using 10 fold cross validation. The author's recognized
    the difficulty of boot strapping the model learning especially in the sit-to
    stand case and trained that model "off policy". They also specifically
    included in the database data from when the system is in the wrong state and
    needs to recover.


    \item \papertitle{Continuous locomotion-mode identification for prosthetic legs based on neuromuscular--mechanical fusion}{huang2011continuous}{-6ex}. 

    \takeaway{Combining EMG and Mechanical data reduces classification error of
    walking modes.} 

    \takeaway{SVMs achieved better performance than linear discriminant
    analysis}

    The author's trained SVM and LDA classifiers to detect gait mode and
    transitions. Features used were EMG time domain features and mechanical
    features (GRF and 6dof load cell data). Fusing EMG + mech worked better than
    just EMG data and SVM worked better (both in terms of accuracy and speed)
    than LDA because of nonlinear decision boundaries (exponential kernel).
    Classification error was computed offline not online via LOOCV.

    \item \papertitle{Intent recognition in a powered lower limb prosthesis
    using time history information}{young2014intent}{-3ex}. 

    \takeaway{A dynamic Bayesian Network ahceived lower LVOO classification error
    than maximum likelihood methods.} 

    \takeaway{This paper treats the problem as a filtering problem isntead of a
    policy learning problem} 

    The authors trained a dynamic Bayesian Network (DBN) to classify walking
    mode and compared classification error rates to maximum likelihood methods
    (LDA) and LDA with majority vote over closely spaced windows. They found the
    produced lower LVOO error rates. The classifier were trained on mechanical
    data only.

    The DBN is similar to a hidden markov model where the current state is
    dependent on the last state and the current observation
    \begin{align}
        \prob{C_t| x_t} &= \frac{\prob{\vec{x}_t|C_t}
        \prob{C_t}}{\prob{\vec{x_t}}} \\
        \prob{C_t} &= \phi \prob{C_{t-1} | x_{t-1}}
    \end{align}
    where $\phi$ is a matrix of transition probabilities. The likelihood,
    $\prob{x|C}$, was assumed to be a multivariate Gaussian.

    This work uses LVOO classification error and doesnt address the DAGGER
    problem but points it out. The follow up work \citep{hargrove2015intuitive}
    is the first to quantify online error.

    The state representation in this paper is just the high level states
    (walking, standing etc). There is no lower level sensor data such as leg
    angle. The states don't seem like they would satisfy Markov property well
    \ie we wont be able to predict a distribution of all future states and
    observations from the current state. Also, the observation models would make
    it equally likely to observe sensor values regardless of gait phase.
    Splitting the walking states in half or thirds based on leg angle to
    represent early/late or early/mid/late stance/swing could make it more
    accurate. This will likely be especially important for Swing/Stance/Trip
    detection. Maybe it would be better to go back to imitation learning and use
    a classification-based policy on sensor data so we can run DAGGER.

    It seems like treating the issue as a filtering problem instead of an policy
    learning problem doesn't allow us to analyize mismatches between prosthesis
    state and human state. For example, It is not necessarily true that the
    prosthesis should be in the MAP state of the human. Example if not correctly
    recognizing a trip is costly we may want to switch to trip recovery even if
    its not the most likely state. Maybe we should use a simple POMDP.

    \takeaway{Definition of state doesn't seem like it comes close to satisfying
    markov property. Also include discretized leg angle?}

    \item \papertitle{A training method for locomotion mode prediction using
    powered lower limb prostheses}{young2014training}{-3ex}. 

    \takeaway{Training data that includes transitions improves the
    LVOO classification error of the transitional steps espeically} 

    The author had subjects walk on the prosthesis through different types of
    gaits. At the transition between gaits, the experimenter indicated the
    transition occured via a GUI\@. LDA classifiers were trained on mechanical
    data to recognize gait mode including the transitions in the training set
    helped improve the error on transitional steps from 90\% to 20\% although it
    also increased error in the steady-state steps from 1\% to 4\%. They
    suggest improving performance by adding EMG and DBN which is finally done
    in~\citet{hargrove2015intuitive} but DAGGER is likely important too.

    \item \papertitle{Intuitive control of a powered prosthetic leg during
    ambulation: a randomized clinical trial}{hargrove2015intuitive}{-6ex}. 
    
    \takeaway{Combining EMG and Mechanical data reduces classification error of
    walking modes.} 

    \takeaway{Dynamic Bayesian Networks (DBN) achieved better performance than
    linear discriminant analysis}

    \takeaway{Offline error is worse than Online error. Possible to solve using
    DAGGER?}
    
    In this paper the researchers collected data sensor and EMG data 
    from 7 transfemoral amputees. They trained classifiers to detect what 
    ambulation mode the user intended to use and transitions between modes.
    Using EMG data as well as mechanical data from the prosthesis reduced
    classification error by $6.2\%$. Offline error computed after the fact
    (Leave One Out) was less than online/realtime error. The author's
    attribute this to the classic reenforcment learning problem:
    Misclassification causes the prosthesis/human to enter states than are not
    in the training data set leading to further degredation of performance.
    This may be solveable using DAGGER~\citet{ross2011reduction}.

\end{enumerate}

\chapter{Trip Recovery} \label{ch:trip_recovery}

\section{Amputee Trip Recovery} \label{sec:amputee_trip_recovery}

\begin{enumerate} \item \papertitle{Transfemoral amputee recovery strategies
following trips to their sound and prosthesis sides throughout swing
phase}{shirota2015transfemoral}{-6ex}. 
    
    \takeaway{Stance side behavior possibly more important than swing behavior
    for trip recovery.} 

    \takeaway{Amputees can still implement all able-bodied
    strategies including elevating strategies}
    
    In this paper the researchers tripped amputees on both their healthy and
    prosthesis sides and tripped healthy subjects on both sides using a
    cable-driven tripping machine. They identified four strategies for able
    bodied subjects: 
    
    \begin{description} 
        \item[Elevating strategy] Subject raised foot and placed it ahead of
        arrested location.  

        \item[Delayed-lowering strategy] Subject raised foot and put it behind
        arresting location 

        \item[Lowering strategy] Subject did not raise foot and placed it at or
        behind arresting location 

        \item[Incomplete Arrest] Subject did not raise foot and placed it ahead
        of arrested location
    \end{description}
    
    Additionally the authors identified two amputee-only strategies:
    \begin{description} 
        \item[Hopping strategy] Subject jumped over virtual obstacle with both
        feet.  

        \item[Skipping strategy] Subject lowered tripped foot and then took an
        extra step with the tripped foot over the obstacle.
    \end{description}
    
    Most individual amputees used a subset of the able-bodied strategies. But
    across all subjects, all able-bodied strategies were utilized. Surprisingly,
    foot kinematics were not different between amputees and healthy-subjects.
    \ie amputees were able to implement all strategies even on the prosthesis
    via sound-side leg and hip movements.
    
    However, the distribution of strategies is different especially when the
    sound side is tripped (amputees used elevating and delayed lowering
    strategies less and used them even less on the sound side). The author's
    credit the large change in the sound side to the lack of active torque on
    the prosthesis side when it is the stance leg. The prosthesis cannot produce
    counteracting moments to halt forward momentum.  
    
\end{enumerate}

\section{Trip Classification} 

Trip Classification papers successfully detect trips and classify them based on
strategy. So far these studies gather data from able-bodied subjects. These
results may not be applicable to amputee subjects.
However,~\citet{shirota2015transfemoral} note the similarity in kinematics of
trip recoveries between amputees and able-bodied subjects so they still might
work.  The main issue is likely the DAGGER problem: data was collected from
subjects who are correctly performing trip recovery. The distribution of states
on a prosthesis who's trip recovery controller makes mistakes will be different
than that of an able-bodied subject. Therefore, the IID assumption is violated
and this should not be treated as a pure supervised learning problem.

\begin{enumerate} 
    \item \papertitle{Stumble detection and classification for an intelligent
    transfemoral prosthesis}{lawson2010stumble}{-3ex}. 

    \takeaway{Successfully implemented trip detection with average delay of 70 ms.}
   
    Collected data from healthy subjects and trained trip detection and
    classification into lowering and elevating strategies. Used FFT features
    in windows of hip, knee, and ankle data of tripped leg. The author's trained
    and tested on the same data so the results are suspect.

    \item \papertitle{Recovery strategy identification throughout swing phase
    using kinematic data from the tripped leg}{shirota2014recovery}{-6ex}. 

    \takeaway{Two part classifier that first classifies walking from trip and
    then the type of trip works better than one part classifier.}
    
    Researcher tripped able-bodied subjects and learned one and two part
    classifiers to classify trip and type of trip. Two part classifier worked
    better. Classifier was learned for each specific subject. Features used were
    min, max, mean, and standard deviation of knee, ankle, foot angle/position
    and acceleration. Features were calculated within windows. Optimal number of
    windows and window length were identified.

\end{enumerate}
%%

\chapter{Policy Learning} \label{ch:Policy Learning}

\section{Learning from Demonstration} \label{sec:Learning from Demonstration}

\begin{enumerate} 

    \item \papertitle{A survey of robot learning from
    demonstration}{argall2009survey}{-3ex}

    This paper organizes the existing learning from demonstration literature
    according to several characteristics:

    \begin{description} 
        \item[Data Gathering] 
            Data gathering strategies attest to deal with the two important
            data correspondence mappings:
            \begin{description} 
                \item[The Record Mapping] 
                    The mapping from the states/actions the teacher experiences
                    to the states/actions recorded in the dataset. For example
                    if learning from able-bodied subjects outfit with Vicon
                    markers, we directly record their joint angles where as if
                    we record the subjects with a camera there is an indirect
                    mapping.
                \item[The Embodiment Mapping] 
                    Whether the states/actions for the system are directly
                    recorded in the dataset. For example, if we can directly use
                    the human's joint angles on the prosthesis, the embodiment
                    mapping is identity. If we want to learn torque commands,
                    then the embodiment mapping would be running inverse
                    dynamics on the recorded joint angles to get the required
                    torques for the prosthesis to follow the recordings from the
                    teacher.
            \end{description} 

            Based on the whether embodiment mapping is identity or not we split
            the data acquisition approaches into two groups:
            \begin{description} 
                \item[Demonstration] 
                    Learning from Demonstration is possible when the embodiment
                    mapping is identity. We further classify strategies in this
                    category based on the record mapping.
                    \begin{description} 
                        \item[Teleoperation] 
                            Used when the record mapping is identity. Examples
                            include demonstrating how a manipulator should move
                            by manually moving it while it is in gravity comp
                            mode or remote controlling a car or plane. 
                        \item[Shadowing] 
                            Used in the case when the record mapping is not
                            identity; the robot senses state from its own
                            sensors while mimicking the teacher.  For example, a
                            robot follows a leader through a maze.
                    \end{description} 
                \item[Imitation] 
                    We employ imitation learning when the embodiment mapping is
                    not identity. We further classify strategies in this
                    category based on the record mapping.
                    \begin{description} 
                        \item[Sensors on Teacher] 
                            Used to make the record mapping direct. Example
                            Vicon markers on a subject to directly record their
                            joint angles. However, the embodiment mapping is
                            not identity so we have to translate the teacher's
                            joint angles to the robot's.
                        \item[External Observation] 
                            When the record mapping is indirect.\ examples we
                            watch the subject via a camera and need to do more
                            complex processing to extract the teacher's state.
                    \end{description} 
            \end{description} 
        \item[Policy Derivation]
            Given data recorded with one of the four possible strategies listed
            above, we learn a policy using one of three general methods:
            \begin{description}
                \item[Mapping Function] 
                    In the mapping function approach we directly learn a policy
                    of the form $\pi: S \rightarrow A$.
                    \begin{description}
                        \item[Classification]
                            If actions and/or states are discrete we can use
                            classification approaches such as Gaussian Mixture
                            Models (GMM), k-Nearest-Neighbors (kNN), logistic
                            regression/SVM, and Hidden Markov Models
                            (HMM).
                        \item[Regression]
                            If states and actions are continuous we use
                            regression such as Locally Weighted Regression
                            (LWR) or Neural Networks (NN).
                    \end{description}
                \item[System Model] 
                    In the system model approach the data is used to learn a
                    probabilistic model of the system dynamics $T(s'| s, a)$ and
                    a reward function $R(S, A)$ is either learned or provided.
                    We use these two to derive a policy $pi: S \rightarrow A$. 
                    \begin{description}
                        \item[Engineered Reward Functions]  
                            The reward function is given. Usually it's very
                            sparse only giving reward at the goal. Rewards can
                            be hard to design manually.
                        \item[Learned reward functions]
                            Use the data to learn the teacher's reward function.
                            Then use the Dynamics model to create a policy.
                            Examples include maximum margin planning and maximum
                            entropy reward functions.
                    \end{description}
                \item[Plans] 
                    Represent the policy as a sequence of actions from state
                    state to goal state. There are preconditions: the state you
                    have to be in before we can take the action and post
                    conditions the state we end up in after taking the action:
                    formal logic controllers.
            \end{description}
    \end{description}

    \item \papertitle{A Reduction of Imitation Learning and Structured
    Prediction to No-Regret Online Learning}{ross2011reduction}{-3ex}. 
    
    \takeaway{To improve imitation learning performance, you should iteratively
    add training data on the states induced by learned polices} 
    
    Imitation learning can lead to poor performance in practice because it
    validates i.i.d.\ assumptions. When the policy makes a mistake, it can lead
    to states not seen in the training data, leading to further mistakes and a
    compounding of errors. The solution is to obtain expert demonstrations of
    actions to take in the states induced by the last learned policy and add
    those to the training dataset.

    The example experiments point to how we could learn trip recovery on the
    prosthesis. In their, experiments, the initial policy is simply the expert
    executing the control. For example, the expert driving the kart. In our
    case, we could start with a hand-tuned policy, or a policy learned from
    gathering data on healthy subjects. Then, we could execute this policy on
    the prosthesis and manually label what strategy should have been used, by
    for example labeling the frames of a video that are synced to the prosthesi
    s data.

    \item \papertitle{Robust trajectory learning and approximation for robot
    programming by demonstration}{aleotti2006robust}{-6ex}. 
    
    \takeaway{Clustering can help group similar demonstrations that represent
    different solutions to a problem.} 
    
    This work falls into the sensors on teacher imitation learning paradigm. The
    authors create an interactive system for programming a manipulator. The
    teacher wears a glove that is tracked by a motion tracking system providing
    demonstrations of end-effector trajectories. The teacher provides multiple
    demonstrations of the same pick and place task.  These demonstrations are
    clustered by an unsupervised clustering algorithm.  An HMM is then trained
    that is used to then rank the demonstrated trajectories within a cluster.
    NURBS splines are then used to fit the highest ranked trajectories within
    each cluster. The fitted NURBS trajectories are shown to the user who can
    then edit them via a GUI. The chosen trajectory is then executed on the
    robot using inverse kinematics (non-identity embodiment mapping). The
    approach does not seem to have much generalizability capability as they don't
    learn a policy.

    \item \papertitle{Incremental learning of gestures by imitation in a
    humanoid robot}{calinon2007incremental}{-3ex}. 
    
    \takeaway{Sensors on teacher + kinesthetic teaching allows the robot to
    learn natural motions and also allows the teacher to refine the motion to
    account for the embodiment mismatch} 

    In this paper, the authors train a small humanoid robot to perform gestures.
    The teacher wears a motion capture system that records his joint angles.
    The trajectories are ran through PCA to reduce their dimensionality. Then
    Gaussian mixture models are trained to fit the trajectories using the EM
    algorithm. Two methods are analyzed to incrementally update the GMM as new
    data is acquired. Sensors on teacher is not solely capable of learning
    policies because of the embodiment mismatch. Kinaesthetic teaching helps
    refine the policy as it is executed like DAGGER. However, kinaesthetic
    teaching alone tends to provide unnatural demonstrations as the teacher can
    not move all joints at once. GMM's provide a convenient way to encode
    trajectories although temporal Gaussian processes seem to make more sense
    for this.

    \item \papertitle{Learning from demonstration and adaptation of biped
    locomotion}{nakanishi2004learning}{-3ex}. 
    
    \takeaway{We can train CPGs using data gathered from biomechanical data but
    the learned policies do not necessarily work well on hardware.} 

    \takeaway{We can also train the CPG on a hand-designed controller to help
    which may improve upon the initial policy.} 

    In this work, the authors train a complex CPG first using biomechanical
    data. The CPG encodes a kinematic trajectory with coupling between joints
    and phase resetting at heel strike. The learned trajectory worked on a
    simulated robot and was robust to disturbances. However, when applied to the
    real robot, the learned trajectory did not work. Instead, they trained the
    CPG on a trajectory generated by a hand-designed policy in order to
    successfully train the CPG.

    \item \papertitle{Movement imitation with nonlinear dynamical systems in
    humanoid robots}{ijspeert2002movement}{-3ex}. 
    
    \takeaway{We can generalize demonstrations to new situations by representing
    the demonstrations as the solutions to differential equations that are
    attracted to the goal state.} 

    This work fits nonlinear differential equations (DMPs) to the demonstrated
    trajectories that are attracted to the goal state. This representation
    appears to have some generalizability as the goal state can be changed in
    the equations to perform different tasks unlike
    in~\citet{aleotti2006robust}. However, some clustering is still required as
    the author's note the policies generalize better when the goal state is
    moved vertically more so than when they are moved horizontally. It's not
    clear why one would use DMPs instead of learning a reward function and then
    optimizing it. The parameterization of the DMP is very unintuitive.
\end{enumerate}


\backmatter

\bibliography{references}
\bibliographystyle{plainnat}

\printindex

\end{document}

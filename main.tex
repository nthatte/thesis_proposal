%%%%%%%%%%%%%%%%%%%%%%%%%%%%%%%%%%%%%%%%%%%%%%%%%%%%%%%%%%%%%%%%%%%%%%
% How to use writeLaTeX: 
%
% You edit the source code here on the left, and the preview on the
% right shows you the result within a few seconds.
%
% Bookmark this page and share the URL with your co-authors. They can
% edit at the same time!
%
% You can upload figures, bibliographies, custom classes and
% styles using the files menu.
%
% If you're new to LaTeX, the wikibook is a great place to start:
% http://en.wikibooks.org/wiki/LaTeX
%
%%%%%%%%%%%%%%%%%%%%%%%%%%%%%%%%%%%%%%%%%%%%%%%%%%%%%%%%%%%%%%%%%%%%%%
\documentclass{tufte-book}

%\hypersetup{colorlinks}% uncomment this line if you prefer colored hyperlinks (e.g., for onscreen viewing)


%%
% If they're installed, use Bergamo and Chantilly from www.fontsite.com.
% They're clones of Bembo and Gill Sans, respectively.
%\IfFileExists{bergamo.sty}{\usepackage[osf]{bergamo}}{}% Bembo
%\IfFileExists{chantill.sty}{\usepackage{chantill}}{}% Gill Sans

\usepackage{microtype}
%%
% For nicely typeset tabular material
\usepackage{booktabs}

%%
% For graphics / images
\usepackage{graphicx}
\setkeys{Gin}{width=\linewidth,totalheight=\textheight,keepaspectratio}
\graphicspath{{graphics/}}

% The fancyvrb package lets us customize the formatting of verbatim
% environments.  We use a slightly smaller font.
\usepackage{fancyvrb}
\fvset{fontsize=\normalsize}

\usepackage{amsmath} % assumes amsmath package installed
\usepackage{amssymb}  % assumes amsmath package installed
%%
% Prints argument within hanging parentheses (i.e., parentheses that take
% up no horizontal space).  Useful in tabular environments.
\newcommand{\hangp}[1]{\makebox[0pt][r]{(}#1\makebox[0pt][l]{)}}

%%
% Prints an asterisk that takes up no horizontal space.
% Useful in tabular environments.
\newcommand{\hangstar}{\makebox[0pt][l]{*}}

%%
% Prints a trailing space in a smart way.
\usepackage{xspace}

%%
% Some shortcuts for Tufte's book titles.  The lowercase commands will
% produce the initials of the book title in italics.  The all-caps commands
% will print out the full title of the book in italics.

% Prints the month name (e.g., January) and the year (e.g., 2008)
\newcommand{\monthyear}{%
  \ifcase\month\or January\or February\or March\or April\or May\or June\or
  July\or August\or September\or October\or November\or
  December\fi\space\number\year
}

% Prints an epigraph and speaker in sans serif, all-caps type.
\newcommand{\openepigraph}[2]{%
  %\sffamily\fontsize{14}{16}\selectfont
  \begin{fullwidth}
  \sffamily\large
  \begin{doublespace}
  \noindent\allcaps{#1}\\% epigraph
  \noindent\allcaps{#2}% author
  \end{doublespace}
  \end{fullwidth}
}

% Inserts a blank page
\newcommand{\blankpage}{\newpage\hbox{}\thispagestyle{empty}\newpage}
\usepackage{units}

% Macros for typesetting the documentation
\newcommand{\hlred}[1]{\textcolor{Maroon}{#1}}% prints in red
\newcommand{\hangleft}[1]{\makebox[0pt][r]{#1}}
\newcommand{\hairsp}{\hspace{1pt}}% hair space
\newcommand{\hquad}{\hskip0.5em\relax}% half quad space
\newcommand{\TODO}{\textcolor{red}{\bf TODO!}\xspace}
\newcommand{\ie}{\textit{i.\hairsp{}e.}\xspace}
\newcommand{\eg}{\textit{e.\hairsp{}g.}\xspace}
\newcommand{\na}{\quad--}% used in tables for N/A cells

%my commands
\synctex=1
\usepackage{comment}
\usepackage{todonotes}
\usepackage[nameinlink]{cleveref}
\usepackage{caption}
\usepackage{subcaption}
\captionsetup{compatibility=false}

\newcommand{\proposaltitle}{Neuromuscular Prosthesis Control for Robust and Natural Gait in Tansfemoral Amputees}

\newcommand{\prob}[1]{\mathrm{P} \left( #1 \right)}
\newcommand{\func}[2]{\mathrm{#1}\left( #2 \right)}

%%
% Book metadata
\title{\proposaltitle}
\author[Nitish Thatte]{Nitish Thatte}
%\publisher{Publisher of This Book}

% Generates the index
\usepackage{makeidx}
\makeindex

\begin{document}

% Front matter
%\frontmatter

% r.3 full title page
%\maketitle
\begin{titlepage}
	\begin{fullwidth}
	\centering
    \phantom.
    \vspace{0.5in}
    {\Large \it Thesis Proposal} \\
    {\huge{\proposaltitle}\par}
    \vspace{0.5in}
    
    Nitish Thatte \\
    \today \\
    \vspace{0.9 in}
    
    The Robotics Institute \\
    Carnegie Mellon University \\
    Pittsburgh, PA 15213
    \vspace{0.9 in}
    
   	{\it Thesis Committee:}\\
    Hartmut Geyer (Chair)\\
    Steven Collins\\
    Chris Atkson\\
    Elliot Rouse, Northwestern University\\
    \vspace{0.9 in}
   
   	{\it Submitted in partial fulfillment of the requirements\\ for the degree of Doctor of Philosophy}\\
    \vspace{0.9 in}
    
    Copyright \copyright \the\year \ Nitish Thatte.
 	\end{fullwidth}
\end{titlepage}

\chapter*{Abstract}
the abstract

% r.5 contents
\tableofcontents

\listoffigures

\listoftables

% r.7 dedication
%\cleardoublepage
%~\vfill
%\begin{doublespace}
%\noindent\fontsize{18}{22}\selectfont\itshape
%\nohyphenation
%Dedicated to those who appreciate \LaTeX{} 
%and the work of \mbox{Edward R.~Tufte} 
%and \mbox{Donald E.~Knuth}.
%\end{doublespace}
%\vfill
%\vfill

%%
% Start the main matter (normal chapters)
\mainmatter
\chapter{Introduction}

\section{Motivation}
\begin{marginfigure}[0.8in]
    \centering
	\begin{subfigure}[b]{\textwidth}
    	\centering
        %\includegraphics{}
        \missingfigure{C-Leg}
        \caption{Ottobock C-Leg knee prosthesis}
        \label{fig:ottobock_cleg}
        \vspace{0.25in}
	\end{subfigure}
	\begin{subfigure}[b]{\textwidth}
    	\centering
        %\includegraphics{}
        \missingfigure{Rheo Knee}
        \caption{Össur Rheo Knee}
        \label{fig:ossur_rheo}
        \vspace{0.25in}
	\end{subfigure}
	\begin{subfigure}[b]{\textwidth}
    	\centering
        %\includegraphics{}
        \missingfigure{Foot}
        \caption{Freedom Innovations Foot}
        \label{fig:freedom_innovations_foot}
	\end{subfigure}
    \caption{Examples of microprocessor-controlled mechanically-passive knee
    prostheses (a,b) and a energy storage and return ankle-foot prosthesis (c).}
\end{marginfigure}
There are currently an estimated six hundred thousand lower-limb amputees in the
United States \citep{ziegler2008estimating}. People suffer from amputations due
to traumatic injuries workplace accidents, traffic collisions, and as casualties
of war. However, a large percentage (54\%) suffer from the loss of a limb due to
complications arising from dysvascular disease associated with diabetes.
Consequently, largely due to the expected increase in diabetes among the
population, \citet{ziegler2008estimating} estimate that by 2050 the number of
amputees living in the United States will likely double.

Currently, transfemoral amputees (those with amputations between the hip and
knee joints) are often prescribed a microprocessor-controlled
mechanically-passive knee prostheses along with an energy storage and return
composite foot such as the Sierra Foot (Freedom Innovations;
Irvine, CA; \cref{fig:freedom_innovations_foot}). The microprocessor knee
prostheses feature control algorithms that measure kinematic and force data via
sensors embedded in the prosthesis and adjust the knee's resistance accordingly.
Examples of microprocessor-controlled prosthetic knees include the C-Leg (Otto
Bock; Duderstadt, Germany; \cref{fig:ottobock_cleg}), which has an adjustable
hydraulic damping system, and the Rheo Knee (Össur; Reykjavik, Iceland;
\cref{fig:ossur_rheo}), which features a variable damping system that uses
magnetorheological fluid. While \citet{johansson2005clinical} show these
microprocessor-controlled knees can improve amputee gait characteristics such as
metabolic energy consumption, peak hip torque, and gait smoothness over those
provided by fully-passive knee prosthesis, these prostheses still cannot fully
replicate healthy leg behavior as they are incapable of providing positive power
during the gait cycle. 

Positive power at the knee is evident in a number of locomotion tasks including
level walking \citep{perry1992gait}, walking up stairs
\citep{nadeau2003frontal}, running \citep{buczek1990stance}, and jumping
\citep{hubley1983work}. In addition, active knee flexion and extension muscle
activations have been noted during stumble recovery \citep{eng1994strategies}.
At the ankle as well, passive spring-like prostheses cannot replicate the
positive net work seen in the ankle joint during level ground walking, which is
essential for push-off and forward propulsion \citep{perry1992gait}.


Consequently, lower-limb amputees, and especially transfemoral or above-the-knee
amputees, equipped with mechanically-passive prostheses suffer from a number of
issues including markedly increased energy consumption~\citep{waters1976energy},
abnormal gait kinematics~\citep{jaegers1995prosthetic}, and an increased
likelihood of falling~\citep{miller2001prevalence}. Specifically, large
percentages of transfemoral amputees report they are unable to complete tasks
such as walking outside in inclement weather (47.4\%), walking while carrying a
load (42.7\%), walking up or down stairs without a handrail (38.5\%, 37.9\%),
walking outside on uneven terrain (29.5\%), picking up an object from the ground
(28.1\%) or getting up from the floor after a fall (22.8\%)
\citep{gauthier1999enabling}.

Importantly, these gait pathologies can lead an avoidance of walking
\citep{gauthier1999enabling}. This is especially true in the case of falls.
\citet{miller2001prevalence} find 49.2\% of lower limb amputees feared falling
and that of those afraid of falls 76\% avoided physical activity as a result.
Avoidance of physical activity is eminently concerning as it may lead to reduced
strength, endurance, and balance, feeding a positive feedback loop that causes
further debilitation.

\begin{marginfigure}
    \centering
    %\includegraphics{}
    \missingfigure{Biom Ankle}
    \caption{Biom Robotic Ankle Prosthesis}
    \label{fig:biom_ankle}
\end{marginfigure}

\begin{figure*}[b]
    \centering
	\begin{subfigure}[b]{0.2\textwidth}
    	\centering
        %\includegraphics{}
        \missingfigure{Gen 1}
        \caption{Generation 1}
	\end{subfigure}
	\begin{subfigure}[b]{0.2\textwidth}
    	\centering
        %\includegraphics{}
        \missingfigure{Gen 2}
        \caption{Generation 2}
	\end{subfigure}
	\begin{subfigure}[b]{0.2\textwidth}
    	\centering
        %\includegraphics{}
        \missingfigure{Gen 3}
        \caption{Generation 3}
	\end{subfigure}
	\begin{subfigure}[b]{0.2\textwidth}
    	\centering
        %\includegraphics{}
        \missingfigure{Gen 4}
        \caption{Generation 4}
	\end{subfigure}
    \caption{Vanderbilt University's Robotic Transfemoral
    Prostheses.\vspace{0.1in}}
    \label{fig:vanderbilt_prostheses}
\end{figure*}


To help remedy this situation, in the past decade researchers and companies have
developed robotic powered knee and ankle prostheses for lower-limb amputees.
These prostheses feature actuators at the knee and/or ankle that, if controlled
correctly, could potentially to restore the kinetics, kinematics, and reactions
of the healthy human leg. Notable examples include four generations of
transfemoral prostheses developed by Vanderbilt
University~(\cref{fig:vanderbilt_prostheses}) \citep{sup2007design,
sup2009preliminary, lawson2013control, lawson2014robotic} and the Biom powered
ankle~(\cref{fig:biom_ankle}) \citep{herr2012bionic}. These powered prostheses
have helped amputees walk on level ground more naturally and efficiently, as
well as walk up stairs and slopes \citep{sup2011upslope, lawson2013control}, run
\citep{huff2012running, shultz2015running}, perform sit-to-stand
\citep{varol2009powered}, and dance \citep{rouse2015design}. These results
illustrate the benefits of powered prostheses as many of these tasks would be
impossible to perform with mechanically-passive prostheses as they require
positive joint power.

\subsection{Challenges in Transfemoral Prosthesis Control}
However, it still remains an open research question how best to control these
prostheses to achieve natural and robust gaits. In the most established control
method for powered prostheses, the prosthesis uses simple impedance functions to
approximate the joint torque versus angle relationships observed during
walking~\citep{sup2009preliminary}. However, since the torque functions only
approximate steady, level walking, this method does not seem to generalize well
to other situations such as walking on slopes~\citep{sup2011upslope} or rough
ground~\citep{thatte2016toward} and changing foot placement
targets~\citep{schepelmann2016evaluation}. 

An alternative approach to joint control in prostheses is to mimic the
underlying dynamics and control of the human neuromuscular system. Rather than
replicating recorded torque profiles with impedance functions, modeling the
dynamical system that generates these torques may help generalize the control to
unexpected situations. Prior work on neuromuscular models shows that they can
lead to robust and natural-looking gaits when used to control a simulated biped.
For example, using a neuromuscular model, an optimized simulated biped model
walked on unseen, uneven terrain with sudden drops and steps up to 14
centimeters~\cite{song2015neural}. In addition, \citet{eilenberg2010control}
successfully applied the neuromuscular control approach to a powered ankle
prosthesis, which mimics the kinematics and kinetics of the ankle joint in human
walking including its adaptation to sloped environments. It remains unclear,
however, whether we can generalize the approach to the more complex function of
the knee joint in gait and balance recovery.

\subsection{Approach}
\section{Expected Contributions}

\begin{enumerate}
    \item Significance of problem
    \begin{enumerate}
        \item Number of amputees and cause of amputations
        \item Amputees face problems due to energy expenditure, unnatural gait,
            fear of falling gait
    \end{enumerate}
    \item Caused by mechanically passive prostheses. The tasks that we need to
    do: regular walking, upslope, downslope, stairs, tripping, stumbling require
    positive energy consumption

    \item People have tried to fix this with active prostheses over the past few
    years:
    \begin{enumerate}
        \item vanderbilt prostheses: slope walking, upstairs, other examples
        \item biom - upslope adaptation, biom dancing
    \end{enumerate}

    \item Control issue with impedance control.  Need many specialized
    controllers. Example different gains for slopes, rough ground, different
    target landing angles. Highlights need for more robust control approach.

    \item Discuss the constraints of prosthesis control:
    \begin{enumerate}
        \item decentralized - we do not wish to sensorize the whole body so
            traditional centralized robotics approaches are inapplicable
        \item dynamicism - human locomotion even walking is very dynamic, cop
        goes to edge of foot, straight knee. etc
        \item decentralized control approaches such as simbicon, impedance
        control, neuromuscular control - has already shown its robustness and
        ability to generalize and also produce natural gaits in simulation
    \end{enumerate}

    \item Approach - neuromuscular inspired prosthesis control
    \begin{enumerate}
        \item First test ideas using simulations of amputees before transferring
        to prosthesis. 
        \item Develop prosthesis hardware that can accurately reproduce the
        torques desired by the model
        \item Tune the parameters of the control for individual users using
        preferences and proposed dagger solution for trip recovery
    \end{enumerate}

    \item Expected contributions
    \begin{enumerate}
        \item SEA prosthesis design capable of producing enough torque for trip
        recovery, impact resilient, torque control.
        \item Evaluation of Neuromuscular prosthesis control in terms of
        energetics, kinematics, and preferences.
        \item method for tuning prostheses and other systems using preferences
        \item Dagger tuning method for learning high level trip recovery policy.
    \end{enumerate}
\end{enumerate}


% The back matter contains appendices, bibliographies, indices, glossaries, etc.

\backmatter

\bibliography{references}
\bibliographystyle{plainnat}

\printindex

\end{document}

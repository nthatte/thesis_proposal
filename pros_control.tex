\chapter{Prosthesis Control} \label{ch:pros_control}

\section{EMG Control} \label{sec:emg_pros_control}

\begin{enumerate} 
    \item \papertitle{A Strategy for Identifying Locomotion Modes Using Surface
    Electromyography}{huang2009strategy}{-3ex}. 

    \takeaway{Paper suggests using Linear Discriminant Analysis instead of
    Artificial Neural Networks, combining EMG time domain or autoregressive
    features with kinematics data as in the later work
    \citet{hargrove2015intuitive}}
    
    In this work EMG sensors were used to measure neuromuscular data from 8
    healthy subjects and 2 TF amputees performing different walking modes. The
    authors used Linear Discriminiant Analysis (LDA) and Artificial Neural
    Networks (ANN) trained on time domain (TD), Autoregressive (AR) features,
    and a combination of both (TDAR) to recognize walking mode during 4
    sub-phases of gait. Different subsets of the EMG sensors were tried and
    different. They found that LDA had lower classification error than ANN and
    was simpler. They found using hip muscles in addition to thigh muscles
    helped even though these would probably be more difficult to measure outside
    of lab. Also they found, using only TD or only AR features worked better
    than using both. A window size of 120ms or more is less noisy and small
    increments between window starts allows for faster updates.  Although, later
    work in~\citet{hargrove2015intuitive} uses a weighted average of
    instantaneous features instead of discrete windows. Also the later work adds
    in mechanical data as is suggested by this work to improve performance. This
    work only reports LOOCV offline validation. The later work also includes
    online validation with slightly worse performance. Dagger may help.  

    \item \papertitle{Multiclass real-time intent recognition of a powered lower
    limb prosthesis}{varol2010multiclass}{-3ex}. 

    \takeaway{Using mechanical sensor data (no emg), LDA instead of PCA for
    dimensionality reduction and Gaussian mixture model the authors built a gait
    intent recognition system}

    The authors built a gait recognition system based on mechanical sensor data.
    They used LDA for dimensionality which is similar to logistic regression. ie
    the normal vector for a dividing hyper plane will tell you which dimensions
    are important and which are not. They used a Gaussian mixture model trained
    with EM to fit the classes. They only trained on tested with one subject and
    report error rates using 10 fold cross validation. The author's recognized
    the difficulty of boot strapping the model learning especially in the sit-to
    stand case and trained that model "off policy". They also specifically
    included in the database data from when the system is in the wrong state and
    needs to recover.


    \item \papertitle{Continuous locomotion-mode identification for prosthetic legs based on neuromuscular--mechanical fusion}{huang2011continuous}{-6ex}. 

    \takeaway{Combining EMG and Mechanical data reduces classification error of
    walking modes.} 

    \takeaway{SVMs achieved better performance than linear discriminant
    analysis}

    The author's trained SVM and LDA classifiers to detect gait mode and
    transitions. Features used were EMG time domain features and mechanical
    features (GRF and 6dof load cell data). Fusing EMG + mech worked better than
    just EMG data and SVM worked better (both in terms of accuracy and speed)
    than LDA because of nonlinear decision boundaries (exponential kernel).
    Classification error was computed offline not online via LOOCV.

    \item \papertitle{Intent recognition in a powered lower limb prosthesis
    using time history information}{young2014intent}{-3ex}. 

    \takeaway{A dynamic Bayesian Network ahceived lower LVOO classification error
    than maximum likelihood methods.} 

    \takeaway{This paper treats the problem as a filtering problem isntead of a
    policy learning problem} 

    The authors trained a dynamic Bayesian Network (DBN) to classify walking
    mode and compared classification error rates to maximum likelihood methods
    (LDA) and LDA with majority vote over closely spaced windows. They found the
    produced lower LVOO error rates. The classifier were trained on mechanical
    data only.

    The DBN is similar to a hidden markov model where the current state is
    dependent on the last state and the current observation
    \begin{align}
        \prob{C_t| x_t} &= \frac{\prob{\vec{x}_t|C_t}
        \prob{C_t}}{\prob{\vec{x_t}}} \\
        \prob{C_t} &= \phi \prob{C_{t-1} | x_{t-1}}
    \end{align}
    where $\phi$ is a matrix of transition probabilities. The likelihood,
    $\prob{x|C}$, was assumed to be a multivariate Gaussian.

    This work uses LVOO classification error and doesnt address the DAGGER
    problem but points it out. The follow up work \citep{hargrove2015intuitive}
    is the first to quantify online error.

    The state representation in this paper is just the high level states
    (walking, standing etc). There is no lower level sensor data such as leg
    angle. The states don't seem like they would satisfy Markov property well
    \ie we wont be able to predict a distribution of all future states and
    observations from the current state. Also, the observation models would make
    it equally likely to observe sensor values regardless of gait phase.
    Splitting the walking states in half or thirds based on leg angle to
    represent early/late or early/mid/late stance/swing could make it more
    accurate. This will likely be especially important for Swing/Stance/Trip
    detection. Maybe it would be better to go back to imitation learning and use
    a classification-based policy on sensor data so we can run DAGGER.

    It seems like treating the issue as a filtering problem instead of an policy
    learning problem doesn't allow us to analyize mismatches between prosthesis
    state and human state. For example, It is not necessarily true that the
    prosthesis should be in the MAP state of the human. Example if not correctly
    recognizing a trip is costly we may want to switch to trip recovery even if
    its not the most likely state. Maybe we should use a simple POMDP.

    \takeaway{Definition of state doesn't seem like it comes close to satisfying
    markov property. Also include discretized leg angle?}

    \item \papertitle{A training method for locomotion mode prediction using
    powered lower limb prostheses}{young2014training}{-3ex}. 

    \takeaway{Training data that includes transitions improves the
    LVOO classification error of the transitional steps espeically} 

    The author had subjects walk on the prosthesis through different types of
    gaits. At the transition between gaits, the experimenter indicated the
    transition occured via a GUI\@. LDA classifiers were trained on mechanical
    data to recognize gait mode including the transitions in the training set
    helped improve the error on transitional steps from 90\% to 20\% although it
    also increased error in the steady-state steps from 1\% to 4\%. They
    suggest improving performance by adding EMG and DBN which is finally done
    in~\citet{hargrove2015intuitive} but DAGGER is likely important too.

    \item \papertitle{Intuitive control of a powered prosthetic leg during
    ambulation: a randomized clinical trial}{hargrove2015intuitive}{-6ex}. 
    
    \takeaway{Combining EMG and Mechanical data reduces classification error of
    walking modes.} 

    \takeaway{Dynamic Bayesian Networks (DBN) achieved better performance than
    linear discriminant analysis}

    \takeaway{Offline error is worse than Online error. Possible to solve using
    DAGGER?}
    
    In this paper the researchers collected data sensor and EMG data 
    from 7 transfemoral amputees. They trained classifiers to detect what 
    ambulation mode the user intended to use and transitions between modes.
    Using EMG data as well as mechanical data from the prosthesis reduced
    classification error by $6.2\%$. Offline error computed after the fact
    (Leave One Out) was less than online/realtime error. The author's
    attribute this to the classic reenforcment learning problem:
    Misclassification causes the prosthesis/human to enter states than are not
    in the training data set leading to further degredation of performance.
    This may be solveable using DAGGER~\citet{ross2011reduction}.

\end{enumerate}

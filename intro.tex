\chapter{Introduction}

\section{Motivation}
\begin{marginfigure}[0.8in]
    \centering
	\begin{subfigure}[b]{\textwidth}
    	\centering
        %\includegraphics{}
        \missingfigure{C-Leg}
        \caption{Ottobock C-Leg knee prosthesis}
        \label{fig:ottobock_cleg}
        \vspace{0.25in}
	\end{subfigure}
	\begin{subfigure}[b]{\textwidth}
    	\centering
        %\includegraphics{}
        \missingfigure{Rheo Knee}
        \caption{Össur Rheo Knee}
        \label{fig:ossur_rheo}
        \vspace{0.25in}
	\end{subfigure}
	\begin{subfigure}[b]{\textwidth}
    	\centering
        %\includegraphics{}
        \missingfigure{Foot}
        \caption{Freedom Innovations Foot}
        \label{fig:freedom_innovations_foot}
	\end{subfigure}
    \caption{Examples of microprocessor-controlled mechanically-passive knee
    prostheses (a,b) and a energy storage and return ankle-foot prosthesis (c).}
\end{marginfigure}
There are currently an estimated six hundred thousand lower-limb amputees in the
United States \citep{ziegler2008estimating}. People suffer from amputations due
to traumatic injuries workplace accidents, traffic collisions, and as casualties
of war. However, a large percentage (54\%) suffer from the loss of a limb due to
complications arising from dysvascular disease associated with diabetes.
Consequently, largely due to the expected increase in diabetes among the
population, \citet{ziegler2008estimating} estimate that by 2050 the number of
amputees living in the United States will likely double.

Currently, transfemoral amputees (those with amputations between the hip and
knee joints) are often prescribed a microprocessor-controlled
mechanically-passive knee prostheses along with an energy storage and return
composite foot such as the Sierra Foot (Freedom Innovations;
Irvine, CA; \cref{fig:freedom_innovations_foot}). The microprocessor knee
prostheses feature control algorithms that measure kinematic and force data via
sensors embedded in the prosthesis and adjust the knee's resistance accordingly.
Examples of microprocessor-controlled prosthetic knees include the C-Leg (Otto
Bock; Duderstadt, Germany; \cref{fig:ottobock_cleg}), which has an adjustable
hydraulic damping system, and the Rheo Knee (Össur; Reykjavik, Iceland;
\cref{fig:ossur_rheo}), which features a variable damping system that uses
magnetorheological fluid. While \citet{johansson2005clinical} show these
microprocessor-controlled knees can improve amputee gait characteristics such as
metabolic energy consumption, peak hip torque, and gait smoothness over those
provided by fully-passive knee prosthesis, these prostheses still cannot fully
replicate healthy leg behavior as they are incapable of providing positive power
during the gait cycle. 

Positive power at the knee is evident in a number of locomotion tasks including
level walking \citep{perry1992gait}, walking up stairs
\citep{nadeau2003frontal}, running \citep{buczek1990stance}, and jumping
\citep{hubley1983work}. In addition, active knee flexion and extension muscle
activations have been noted during stumble recovery \citep{eng1994strategies}.
At the ankle as well, passive spring-like prostheses cannot replicate the
positive net work seen in the ankle joint during level ground walking, which is
essential for push-off and forward propulsion \citep{perry1992gait}.

Consequently, lower-limb amputees, and especially transfemoral or above-the-knee
amputees, equipped with mechanically-passive prostheses suffer from a number of
issues including markedly increased energy consumption~\citep{waters1976energy},
abnormal gait kinematics~\citep{jaegers1995prosthetic}, and an increased
likelihood of falling~\citep{miller2001prevalence}. Specifically, large
percentages of transfemoral amputees report they are unable to complete tasks
such as walking outside in inclement weather (47.4\%), walking while carrying a
load (42.7\%), walking up or down stairs without a handrail (38.5\%, 37.9\%),
walking outside on uneven terrain (29.5\%), picking up an object from the ground
(28.1\%) or getting up from the floor after a fall (22.8\%)
\citep{gauthier1999enabling}.

Importantly, these gait pathologies can lead an avoidance of walking
\citep{gauthier1999enabling}. This is especially true in the case of falls.
\citet{miller2001prevalence} find 49.2\% of lower limb amputees feared falling
and that of those afraid of falls 76\% avoided physical activity as a result.
Avoidance of physical activity is eminently concerning as it may lead to reduced
strength, endurance, and balance, feeding a positive feedback loop that causes
further debilitation.

\begin{marginfigure}
    \centering
    %\includegraphics{}
    \missingfigure{Biom Ankle}
    \caption{Biom Robotic Ankle Prosthesis}
    \label{fig:biom_ankle}
\end{marginfigure}

\begin{figure*}[b]
    \centering
	\begin{subfigure}[b]{0.2\textwidth}
    	\centering
        %\includegraphics{}
        \missingfigure{Gen 1}
        \caption{Generation 1}
	\end{subfigure}
	\begin{subfigure}[b]{0.2\textwidth}
    	\centering
        %\includegraphics{}
        \missingfigure{Gen 2}
        \caption{Generation 2}
	\end{subfigure}
	\begin{subfigure}[b]{0.2\textwidth}
    	\centering
        %\includegraphics{}
        \missingfigure{Gen 3}
        \caption{Generation 3}
	\end{subfigure}
	\begin{subfigure}[b]{0.2\textwidth}
    	\centering
        %\includegraphics{}
        \missingfigure{Gen 4}
        \caption{Generation 4}
	\end{subfigure}
    \caption{Vanderbilt University's Robotic Transfemoral
    Prostheses.\vspace{0.1in}}
    \label{fig:vanderbilt_prostheses}
\end{figure*}

To help remedy this situation, in the past decade researchers and companies have
developed robotic powered knee and ankle prostheses for lower-limb amputees.
These prostheses feature actuators at the knee and/or ankle that, if controlled
correctly, could potentially restore the kinetics, kinematics, and reactions
of the healthy human leg. Notable examples include four generations of
transfemoral prostheses developed by Vanderbilt
University~(\cref{fig:vanderbilt_prostheses}) \citep{sup2007design,
sup2009preliminary, lawson2013control, lawson2014robotic} and the Biom powered
ankle~(\cref{fig:biom_ankle}) \citep{herr2012bionic}. These powered prostheses
have helped amputees walk on level ground more naturally and efficiently, as
well as walk up stairs and slopes \citep{sup2011upslope, lawson2013control}, run
\citep{huff2012running, shultz2015running}, perform sit-to-stand
\citep{varol2009powered}, and dance \citep{rouse2015design}. These results
illustrate the benefits of powered prostheses as many of these tasks would be
impossible to perform with mechanically-passive prostheses as they require
positive joint power.

\subsection{Challenges in Transfemoral Prosthesis Control}
It still remains an open research question how best to control these prostheses
to achieve natural and robust gaits. In the most established control method for
powered prostheses, the prosthesis uses simple impedance functions to
approximate the joint torque versus angle relationships observed during
walking~\citep{sup2009preliminary}. However, since the torque functions only
approximate steady, level walking, this method does not seem to generalize well
to other situations such as walking on slopes~\citep{sup2011upslope} or rough
ground~\citep{thatte2016toward} and changing foot placement
targets~\citep{schepelmann2016evaluation}. 

As mentioned earlier, walking on slopes and rough ground present major hurdles
for transfemoral amputees. Moreover, previously developed prosthesis controls
have not specifically addressed the risk of falling that is so detrimental to
amputee quality of life. Therefore, it is clear that we should formulate a
prosthesis controller with more power to generalize to a larger variety of
environments thereby improving amputee gait robustness. Formulating a robotic
prosthesis controller to accomplish this goal requires we address three main
challenges:

\begin{description}
    \item[Challenge 1: Human locomotion is a dynamic task] 
    \begin{marginfigure}[-0.5in]
        \centering
        %\includegraphics{}
        \missingfigure{passive dynamic walker}
        \caption{A passive dynamic walker}
        \label{fig:passive_dynamic_walker}
    \end{marginfigure}
    
    Human locomotion is characterized by dynamic interaction with the
    environment \citep{mcgeer_1992}. During stance, the leg acts in a compliant,
    spring-like manner \citep{geyer2006compliant} and significant time is spent
    in statically-unstable contact on the heel or toe, suggesting the importance
    of mechanical stability achieved via foot placement \citep{perry1992gait}.
    During swing, ballistic motion explains much of the leg trajectory
    \citep{mochon1980ballistic}. Indeed, much of the entire gait cycle can be
    explained via passive dynamics as evidenced by passive-dynamic walkers
    (\cref{fig:passive_dynamic_walker}) that can stably walk down slight
    inclines with no onboard power source~\citep{mcgeer1990passive,
    collins2005efficient}.

    \begin{marginfigure}[-0.5in]
        \centering
        %\includegraphics{}
        \missingfigure{Rheo Knee}
        \caption{Honda's Asimo Robot uses position control and statically
        stable gaits}
        \label{fig:asimo}
    \end{marginfigure}

    Consequently, in order to ensure that amputee gaits are natural and
    efficient, but still robust it is essential that robotic prosthesis
    controllers not only admit, but leverage the inherent dynamics of walking.
    Therefore, the required control paradigm cannot follow strategies often used
    for humanoid locomotion (for example on Honda's Asimo Robot
    \cref{fig:asimo}) that employ position control in order to track preplanned,
    statically-stable gaits. Rather, the control strategy should interact
    dynamically with the amputee by governing interaction forces instead of
    mandating kinematic objectives.

    \item[Challenge 2: We have incomplete state information]

    An additional difference between robotic prosthesis control and controls
    often used on humanoid walking robots stems from the lack of full state
    information. In humanoid walking controllers such as those used in the DARPA
    robotics challenge, controllers use the full state information to
    plan and track a trajectory for robot's center of mass, thereby
    ensuring stability of the full system \citep{feng2015optimization,
    kuindersma2014efficiently, englsberger2014trajectory}.
    
    Unlike a humanoid robot control scenario in which we know the positions,
    velocities, and forces applied to every joint and the center of mass, for
    prosthesis control we typically only know the state of the prosthesis
    itself. It is unreasonable to expect that amputees will don full body
    sensing suits in order to provide a complete picture of the state of the
    amputee-prosthesis system. Therefore, prosthesis controllers must be
    decentralized, meaning joint torque commands are computed using only a
    subset of the full state. A side affect of this approach is a loss of formal
    stability guarantees. However, we can still evaluate amputee stability
    empirically.

    \item[Challenge 3: Amputees are unique]
    
    Finally, we should be able to adapt robotic prosthesis controllers to each
    amputees individual's needs. The variation in amputee needs arise from a
    number of factors including but not limited to the amputee's height, weight,
    strength, endurance, reason for amputation, time since amputation, 
    experience, and personal preferences. Consequently, prostheses and
    controllers should be optimized to suit individual users.
\end{description}

\begin{fullwidth}
\emph{This thesis proposes decentralized, dynamic control methods for
transfemoral prostheses, along with methods to optimize them for individual
patients, in order to improve amputee gait robustness and naturalness.}
\end{fullwidth}

\subsection{Approach}

In this thesis, we seek to improve amputee gait robustness and naturalness by
employing an alternative approach to joint control in prostheses that seeks to
mimic the underlying dynamics and control of the human neuromuscular system. In
this approach, instead of replicating recorded torque profiles with impedance
functions, we model the dynamical system, consisting of virtual muscles and
local reflex feedback pathways, that generate joint torques during locomotion.
Crucially, the resulting prosthesis control is decentralized, as the reflex
feedback are designed to rely only on the state of other muscles in the same
leg. Additionally, it is dynamic, as the virtual muscles integrate the sensed
kinematic state of the prosthesis in order to generate desired torques at the
joints. These torques, along with the reaction forces in the amputee's socket
and on the ground shape the motion of the amputee-prosthesis system.
\begin{marginfigure}
    \centering
    %\includegraphics{}
    \missingfigure{muscle ankle}
    \caption{\citet{eilenberg2010control} simulate virtual muscles in
    order to control an ankle prosthesis.}
    \label{fig:eilenberg_muscle}
\end{marginfigure}

Prior work on neuromuscular models shows that they can lead to robust and
natural-looking gaits when used to control simulated bipeds. For example, using
a neuromuscular model, an optimized simulated biped model walked on unseen,
uneven terrain with sudden drops and steps up to 14 centimeters
\citep{song2015neural}. In addition, \citet{eilenberg2010control} successfully
applied the neuromuscular control approach to a powered ankle prosthesis
(\cref{fig:eilenberg_muscle}), which mimics the kinematics and kinetics of the
ankle joint in human walking including its adaptation to sloped environments. It
remains unclear, however, whether we can extend the approach to transfemoral
prostheses with both knee and ankle joints.

To test this control approach, we will first design and build custom
transfemoral prosthesis hardware. Two important features the prosthesis must
have are the ability to produce enough torque and speed to enable trip recovery
and the ability to faithfully reproduce the desired joint torques commanded by
the neuromuscular model. To achieve these goals, we employ series-elastic
actuation~\citep{pratt1995series} in the proposed design
(\cref{fig:prosthesis_design}), which allows for precise control of joint
torque. Furthermore, series elastic actuators also help protect the prosthesis
from the impact loads it will experience during trip recovery experiments. 

Next, to motivate our specific choice of neuromuscular control for improving
amputee gait stability, we construct a simulation of the amputee-prosthesis
system and compare the gait robustness achieved by neuromuscular control versus
the established impedance control method.

To address the challenge of optimizing prosthesis controls for specific subjects
we develop an algorithm that uses preference feedback from the amputee to
optimize the prosthesis control parameters. We test the method first by
optimizing synthetic reward functions of increasing relevance and by optimizing
neuromuscular control parameters for intact users wearing the prosthesis through
an amputee emulator brace. Finally, we use the method to optimize the control
parameters for a specific amputee wearing and evaluate the resulting prosthesis
control performance relative to the amputee's prescribed prosthesis.

Lastly, in order to further improve the capability of the transfemoral
prosthesis to respond to trips, we seek to use imitation learning techniques
\citep{argall2009survey}, in order to allow the prosthesis to mimic observed
trip recovery strategies employed after disturbances to the swing leg.

\section{Expected Contributions}

Work presented in this thesis will advance the state-of-the-art for robotic
transfemoral prosthesis control and optimization. There are three main expected 
contributions: 

\begin{description}
    \item[Contribution 1: series elastic prosthesis design] We present the
    design of a transfemoral prosthesis featuring series elastic actuators
    (SEAs) capable of accurately producing the torques commanded by the
    neuromuscular model, generating enough torque and speed to enable trip
    recovery experiments, and handling the impact loads expected during trip
    recovery experiments. We have made significant progress towards this
    contribution already by completing the design, manufacturing, assembly, and
    initial testing of the prosthesis' knee joint as well as the design and
    fabrication of its ankle joint. \Cref{fig:prosthesis_design} shows the
    current stage of the prosthesis prototype with the completed SEA knee and a
    passive spring-loaded ankle well as a CAD render of the expected completed
    prosthesis design.
    \begin{marginfigure}
        \centering
        %\includegraphics{}
        \missingfigure{prosthesis}
        \caption{Proposed SEA prosthesis design}
        \label{fig:prosthesis_design}
    \end{marginfigure}

    \item[Contribution 2: evaluation of neuromuscular transfemoral prosthesis control]
    test

    \item[Contribution 3: method for optimizing prostheses via preferences]
    test

    \item[Contribution 4: learning and evaluation of a trip recovery policy]
    test
\end{description}

\begin{enumerate}
    \item Significance of problem
    \begin{enumerate}
        \item Number of amputees and cause of amputations
        \item Amputees face problems due to energy expenditure, unnatural gait,
            fear of falling gait
    \end{enumerate}
    \item Caused by mechanically passive prostheses. The tasks that we need to
    do: regular walking, upslope, downslope, stairs, tripping, stumbling require
    positive energy consumption

    \item People have tried to fix this with active prostheses over the past few
    years:
    \begin{enumerate}
        \item vanderbilt prostheses: slope walking, upstairs, other examples
        \item biom - upslope adaptation, biom dancing
    \end{enumerate}

    \item Control issue with impedance control.  Need many specialized
    controllers. Example different gains for slopes, rough ground, different
    target landing angles. Highlights need for more robust control approach.

    \item Discuss the constraints of prosthesis control:
    \begin{enumerate}
        \item decentralized - we do not wish to sensorize the whole body so
            traditional centralized robotics approaches are inapplicable
        \item dynamicism - human locomotion even walking is very dynamic, cop
        goes to edge of foot, straight knee. etc
        \item decentralized control approaches such as simbicon, impedance
        control, neuromuscular control - has already shown its robustness and
        ability to generalize and also produce natural gaits in simulation
    \end{enumerate}

    \item Approach - neuromuscular inspired prosthesis control
    \begin{enumerate}
        \item First test ideas using simulations of amputees before transferring
        to prosthesis. 
        \item Develop prosthesis hardware that can accurately reproduce the
        torques desired by the model
        \item Tune the parameters of the control for individual users using
        preferences and proposed dagger solution for trip recovery
    \end{enumerate}

    \item Expected contributions
    \begin{enumerate}
        \item SEA prosthesis design capable of producing enough torque for trip
        recovery, impact resilient, torque control.
        \item Evaluation of Neuromuscular prosthesis control in terms of
        energetics, kinematics, and preferences.
        \item method for tuning prostheses and other systems using preferences
        \item Dagger tuning method for learning high level trip recovery policy.
    \end{enumerate}
\end{enumerate}


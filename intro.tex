\chapter{Introduction}

\section{Motivation}
\begin{marginfigure}[0.8in]
    \centering
	\begin{subfigure}[b]{\textwidth}
    	\centering
        %\includegraphics{}
        \missingfigure{C-Leg}
        \caption{Ottobock C-Leg knee prosthesis}
        \label{fig:ottobock_cleg}
        \vspace{0.25in}
	\end{subfigure}
	\begin{subfigure}[b]{\textwidth}
    	\centering
        %\includegraphics{}
        \missingfigure{Rheo Knee}
        \caption{Össur Rheo Knee}
        \label{fig:ossur_rheo}
        \vspace{0.25in}
	\end{subfigure}
	\begin{subfigure}[b]{\textwidth}
    	\centering
        %\includegraphics{}
        \missingfigure{Foot}
        \caption{Freedom Innovations Foot}
        \label{fig:freedom_innovations_foot}
	\end{subfigure}
    \caption{Examples of microprocessor-controlled mechanically-passive knee
    prostheses (a,b) and a energy storage and return ankle-foot prosthesis (c).}
\end{marginfigure}
There are currently an estimated six hundred thousand lower-limb amputees in the
United States \citep{ziegler2008estimating}. People undergo amputations due to a
variety of reasons including traumatic injuries from workplace accidents,
traffic collisions, and as casualties of war. In addition, a large percentage
(54\%) suffer from the loss of a limb due to complications arising from
dysvascular disease associated with diabetes.  Consequently, largely due to the
expected increase in diabetes in the coming years,
\citet{ziegler2008estimating} estimate that by 2050 the number of amputees
living in the United States will likely double.

Currently, prosthetists often prescribe transfemoral amputees (those with
amputations between the hip and knee joints) an energy storage and return
composite foot such as the Sierra Foot (Freedom Innovations; Irvine, CA;
\cref{fig:freedom_innovations_foot}) along with a microprocessor-controlled
mechanically-passive knee prostheses. These knee prostheses feature control
algorithms that measure kinematic and force data via sensors embedded in the
prosthesis and adjust the knee's resistance accordingly.  Examples of
microprocessor-controlled prosthetic knees include the C-Leg (Otto Bock;
Duderstadt, Germany; \cref{fig:ottobock_cleg}), which has an adjustable
hydraulic damping system, and the Rheo Knee (Össur; Reykjavik, Iceland;
\cref{fig:ossur_rheo}), which achieves variable damping via a magnetorheological
fluid. While \citet{johansson2005clinical} show these microprocessor-controlled
knees can improve amputee gait characteristics by decreasing metabolic energy
consumption and peak hip torque and increasing gait smoothness over that
provided by fully-passive knee prosthesis, these prostheses still cannot fully
replicate healthy leg behavior as they are incapable of providing positive power
during the gait cycle. 

Positive power at the knee is evident in a number of locomotion tasks including
level walking \citep{perry1992gait}, walking up stairs
\citep{nadeau2003frontal}, running \citep{buczek1990stance}, and jumping
\citep{hubley1983work}. In addition, active knee flexion and extension muscle
activations have been noted during stumble recovery \citep{eng1994strategies}.
At the ankle as well, passive spring-like prostheses cannot replicate the
positive net work seen in the ankle joint during level ground walking, which is
essential for push-off and forward propulsion \citep{perry1992gait}.

Consequently, lower-limb amputees, and especially transfemoral or above-the-knee
amputees, equipped with mechanically-passive prostheses suffer from a number of
issues including markedly increased energy consumption~\citep{waters1976energy},
abnormal gait kinematics~\citep{jaegers1995prosthetic}, and an increased
likelihood of falling~\citep{miller2001prevalence}. Specifically, large
percentages of transfemoral amputees report they are unable to complete tasks
such as walking outside in inclement weather (47.4\%), walking while carrying a
load (42.7\%), walking up or down stairs without a handrail (38.5\%, 37.9\%),
walking outside on uneven terrain (29.5\%), picking up an object from the ground
(28.1\%) or getting up from the floor after a fall (22.8\%)
\citep{gauthier1999enabling}.

Importantly, these gait pathologies can lead an avoidance of walking
\citep{gauthier1999enabling}. This is especially true in the case of falls.
\citet{miller2001prevalence} find 49.2\% of lower limb amputees feared falling
and that of those afraid of falls 76\% avoided physical activity as a result.
Avoidance of physical activity is eminently concerning as it may lead to reduced
strength, endurance, and balance, feeding a positive feedback loop that causes
further debilitation.
\begin{figure*}[b]
    \centering
	\begin{subfigure}[b]{0.2\textwidth}
    	\centering
        %\includegraphics{}
        \missingfigure{Gen 1}
        \caption{Generation 1}
	\end{subfigure}
	\begin{subfigure}[b]{0.2\textwidth}
    	\centering
        %\includegraphics{}
        \missingfigure{Gen 2}
        \caption{Generation 2}
	\end{subfigure}
	\begin{subfigure}[b]{0.2\textwidth}
    	\centering
        %\includegraphics{}
        \missingfigure{Gen 3}
        \caption{Generation 3}
	\end{subfigure}
	\begin{subfigure}[b]{0.2\textwidth}
    	\centering
        %\includegraphics{}
        \missingfigure{Gen 4}
        \caption{Generation 4}
	\end{subfigure}
    \caption{Vanderbilt University's Robotic Transfemoral
    Prostheses.\vspace{0.1in}}
    \label{fig:vanderbilt_prostheses}
\end{figure*}

To help remedy this situation, in the past decade researchers and companies have
developed robotic powered knee and ankle prostheses for lower-limb amputees.
These prostheses feature actuators at the knee and/or ankle that, if controlled
correctly, could potentially restore the kinetics, kinematics, and reactions
of the healthy human leg. Notable examples include four generations of
transfemoral prostheses developed by Vanderbilt
University~(\cref{fig:vanderbilt_prostheses}) \citep{sup2007design,
sup2009preliminary, lawson2013control, lawson2014robotic} and the Biom powered
ankle~(\cref{fig:biom_ankle}) \citep{herr2012bionic}. These powered prostheses
have helped amputees walk on level ground more naturally and efficiently, as
well as walk up stairs and slopes \citep{sup2011upslope, lawson2013control}, run
\citep{huff2012running, shultz2015running}, perform sit-to-stand
\citep{varol2009powered}, and dance \citep{rouse2015design}. These results
illustrate the benefits of powered prostheses as many of these tasks require
positive joint power and thus would be difficult to perform with
mechanically-passive prostheses.

\begin{marginfigure}[-1in]
    \centering
    %\includegraphics{}
    \missingfigure{Biom Ankle}
    \caption{Biom Robotic Ankle Prosthesis}
    \label{fig:biom_ankle}
\end{marginfigure}


\subsection{Challenges in Transfemoral Prosthesis Control}\label{sec:challenges}
It still remains an open research question how best to control these prostheses
to achieve natural and robust gaits. In the most established control method for
powered prostheses, the prosthesis uses simple impedance functions to
approximate the joint torque versus angle relationships observed during
walking~\citep{sup2009preliminary}. However, since the torque functions only
approximate steady, level walking, this method does not seem to generalize well
to other situations such as walking on slopes~\citep{sup2011upslope} or rough
ground~\citep{thatte2016toward} and changing foot placement
targets~\citep{schepelmann2016evaluation}. 

As mentioned earlier, walking on slopes and rough ground present major hurdles
for transfemoral amputees. Moreover, previously developed prosthesis controls
have not specifically addressed the risk of falling that is so detrimental to
amputee quality of life. Therefore, it is clear that we should formulate a
prosthesis controller with more power to generalize to a larger variety of
environments thereby improving amputee gait robustness. Formulating a robotic
prosthesis controller to accomplish this goal requires we address three main
challenges:

\begin{marginfigure}
    \centering
    %\includegraphics{}
    \missingfigure{passive dynamic walker}
    \caption{A passive dynamic walker walks down hill with no internal
    actuation highlighting the role of natural dynamics in walking.}
    \label{fig:passive_dynamic_walker}
\end{marginfigure}

\paragraph{Challenge 1: Human locomotion is a dynamic task}
Human locomotion is characterized by dynamic interaction with the
environment \citep{mcgeer_1992}. During stance, the leg acts in a compliant,
spring-like manner \citep{geyer2006compliant} and significant time is spent
in statically-unstable contact on the heel or toe, suggesting the importance
of mechanical stability achieved via foot placement \citep{perry1992gait}.
During swing, ballistic motion explains much of the leg trajectory
\citep{mochon1980ballistic}. Indeed, much of the entire gait cycle can be
explained via passive dynamics as evidenced by passive-dynamic walkers
(\cref{fig:passive_dynamic_walker}) that can stably walk down slight
inclines with no onboard power source~\citep{mcgeer1990passive,
collins2005efficient}.

\begin{marginfigure}
    \centering
    %\includegraphics{}
    \missingfigure{Rheo Knee}
    \caption{Honda's Asimo Robot uses position control and statically
    stable gaits.}
    \label{fig:asimo}
\end{marginfigure}

Consequently, in order to ensure that amputee gaits are natural and efficient,
but still robust, it is essential that robotic prosthesis controllers not only
admit, but leverage the inherent dynamics of walking.  Therefore, the required
control paradigm cannot follow strategies often used for humanoid locomotion
(for example on Honda's Asimo Robot \cref{fig:asimo}) that employ position
control in order to track preplanned, statically-stable gaits. Rather, the
control strategy should interact dynamically with the amputee by governing
interaction forces instead of mandating kinematic objectives.

\paragraph{Challenge 2: We have incomplete state information}

An additional difference between robotic prosthesis control and controls often
used on humanoid walking robots stems from the lack of full state information.
Humanoid walking controllers such as those used in the DARPA robotics challenge
(DRC), controllers use the full state of the robot (\ie the positions and
velocities of every joint and the robot's center of mass), to plan and track a
trajectories, thereby ensuring stability of the full system
\citep{feng2015optimization, kuindersma2014efficiently,
englsberger2014trajectory}.

While these recent approaches used in the DRC are dynamic and therefore address
challenge 1, for prosthesis control we typically only know the state of the
prosthesis itself. It is unreasonable to expect that amputees will don full body
sensing suits in order to provide a complete picture of the state of the
amputee-prosthesis system. Therefore, prosthesis controllers must be
decentralized, meaning joint torque commands are computed using only a subset of
the full state. A side affect of this approach is a loss of formal stability
guarantees. However, we can still evaluate amputee stability empirically.

\paragraph{Challenge 3: Amputees are unique}

Finally, we should be able to adapt robotic prosthesis controllers to each
amputees individual's needs. The variation in amputee needs arise from a number
of factors including but not limited to the amputee's height, weight, strength,
endurance, reason for amputation, time since amputation, experience, and
personal preferences. Consequently, prostheses and controllers should be
optimized to suit individual users.

\vspace{2ex}
\begin{fullwidth} \emph{This thesis proposes decentralized, dynamic control
methods for transfemoral prostheses, along with methods to optimize them for
individual amputees, in order to improve gait robustness and naturalness.}
\end{fullwidth}

\subsection{Approach}

In this thesis, we seek to improve amputee gait robustness and naturalness by
employing an alternative approach to joint control in prostheses that seeks to
mimic the underlying dynamics and control of the human neuromuscular system. In
this approach, instead of replicating recorded torque profiles with impedance
functions, we model the dynamical system, consisting of virtual muscles and
local reflex feedback pathways, that generate joint torques during locomotion.
Crucially, the resulting prosthesis control addresses challenges 1 and 2: the
control is decentralized, as the reflex feedback are designed to rely only on
the state of other muscles in the same leg, and dynamic, as the virtual muscles
integrate the sensed kinematic state of the prosthesis in order to generate
desired torques, not positions, at the joints. These torques, along with the
reaction forces in the amputee's socket and on the ground shape the motion of
the amputee-prosthesis system.
\begin{marginfigure}
    \centering
    %\includegraphics{}
    \missingfigure{muscle ankle}
    \caption{\citet{eilenberg2010control} simulate virtual muscles in
    order to control an ankle prosthesis.}
    \label{fig:eilenberg_muscle}
\end{marginfigure}

Prior work on neuromuscular models shows that they can lead to robust and
natural-looking gaits when used to control simulated bipeds. For example, using
a neuromuscular model, an optimized simulated biped model walked on unseen,
uneven terrain with sudden drops and steps up to 14 centimeters
\citep{song2015neural}. In addition, \citet{eilenberg2010control} successfully
applied the neuromuscular control approach to a powered ankle prosthesis
(\cref{fig:eilenberg_muscle}), which mimics the kinematics and kinetics of the
ankle joint in human walking including its adaptation to sloped environments. It
remains unclear, however, whether we can extend the approach to transfemoral
prostheses with both knee and ankle joints.

Therefore, to motivate our specific choice of neuromuscular control for
improving amputee gait stability, in completed work, we construct a simulation
of the amputee-prosthesis system and compare the gait robustness achieved by
neuromuscular control versus the established impedance control method. We find
that neuromuscular control enabled the simulated amputee to walk further over
rougher terrain than the established impedance control method allows.

Next to test the feasibility of the control approach to control a real system, we
design and build a partial powered transfemoral prosthesis prototype with an
active knee actuator and a passive, spring-loaded ankle. The prosthesis
prototype uses series elastic actuation~\citep{pratt1995series} that allows it
to accurately achieve the torques commanded by the neuromuscular model. Initial
tests with an intact user wearing the prosthesis through an amputee simulator
adaptor show that the proposed neuromuscular control when applied to the knee of
an active prosthesis, produce reasonable kinematics and joint torques. This
positive result motivates continued development of the prosthesis into a
full active knee and ankle transfemoral prosthesis and implementation of testing
of the full neuromuscular prosthesis control.

To address the challenge 3 we propose to optimize prosthesis controls for
specific subjects. To this end, in completed work, we develop an algorithm that
uses preference feedback from users to optimize control system parameters. We
test the method on problems of increasing relevance: first by optimizing
synthetic reward functions, then optimizing the parameters of simulated
dynamical systems, and finally by optimizing neuromuscular control parameters
for intact users wearing the prosthesis through an amputee emulator brace. The
results suggest the proposed optimization method outperforms baseline methods
for optimizing from user preferences. However, it remains to be seen if the
proposed method improves gait characteristics when applied to the full
neuromuscular controlled prosthesis for an amputee subject. We intend to
investigate this question via additional tests on an amputee subject.

Lastly, we seek to improve the capability of the transfemoral prosthesis to
respond to trips, which pose a significant and impactful threat to amputee
quality of life. To accomplish this goal, we propose to use imitation learning
techniques \citep{argall2009survey} to learn polices that allow the prosthesis
to appropriately respond to disturbances during swing.  The proposed method to
learn these policies will address challenges 1-3 in that it will be a
decentralized control that only uses information from the prosthesis and be
dynamic and personalized by working with each amputee' innate trip response
reflexes.

Previous work in this area has trained classifiers on data obtained by tripping
healthy human subjects~\citep{lawson2010stumble, shirota2014recovery}. The
authors then evaluate these classifiers via cross validation, in which a subset
of the training data is set aside and used for testing, and report low
error-rates.  However, to date no one has applied a trip classifier to
prosthesis hardware in order to initiate a trip recovery controller. Trivial
application of classifiers trained on healthy human subject data likely will not
work, as the distribution of data at test time generated by a prosthesis that is
controlled by a learned policy will differ from the data used to train that
policy. The training and test time distribution mismatch violates the i.i.d.\
(independent and identically drawn) assumption that underpins classification
performance. To remedy this problem, we intend to employ the DAGGER training
method \citep{ross2011reduction} that aligns the train and test time
distributions through an iterative procedure. We will collect training and
testing data to learn and evaluate the trip recovery policies using the Push Bot
robot (\cref{fig:push_bot}) which can apply tripping forces to subjects via
actuated tethers.

\begin{marginfigure}
    \centering
    %\includegraphics{}
    \missingfigure{push bot}
    \caption{Push Bot robot for training and evaluating trip recovery policies}
    \label{fig:push_bot}
\end{marginfigure}

\section{Expected Contributions}

Work presented in this thesis will advance the state-of-the-art for robotic
transfemoral prosthesis control and optimization. There are four main expected 
contributions: 

\begin{marginfigure}
    \centering
    %\includegraphics{}
    \missingfigure{prosthesis}
    \caption{Proposed SEA prosthesis design}
    \label{fig:prosthesis_design}
\end{marginfigure}
\paragraph{Contribution 1: A series elastic prosthesis design} 
We present the design of a transfemoral prosthesis featuring series elastic
actuators (SEAs) capable of accurately producing the torques commanded by the
neuromuscular model, generating enough torque and speed to enable trip recovery
experiments, and handling the impact loads expected during trip recovery
experiments. We have made significant progress towards this contribution already
by completing the design, manufacturing, assembly, and initial testing of the
prosthesis' knee joint as well as the design and fabrication of its ankle joint.
\Cref{fig:prosthesis_design} shows the current stage of the prosthesis prototype
with the completed SEA knee and a passive spring-loaded ankle well as a CAD
render of the expected completed prosthesis design.

\paragraph{Contribution 2: A method for optimizing systems via preferences} We
present a new algorithm for optimizing systems, such as prostheses, using user
preferences.  The algorithm uses preferences between pairs of control parameters
to circumvent having to define or learn an explicit reward function for each
user. Additionally, the algorithm employs Bayesian optimization techniques in
order to query users for preferences that are expected to maximally reduce the
uncertainty of the location of the optimum parameters.

\paragraph{Contribution 3: Evaluation of neuromuscular transfemoral prosthesis
control} We will implement the proposed neuromuscular prosthesis control on the
SEA transfemoral prosthesis, optimize its parameters according to the amputee
subject's preferences, and evaluate the prosthesis' ability to produce a natural
and comfortable gait. We will measure gait characteristics in terms of joint
kinematics and kinetics and the amputee's metabolic energy consumption. We will
present results relative to typical non-amputee gait, the amputee's gait using
his or her prescribed prosthesis, and the gait achieved by an unoptimized
prosthesis control.

\paragraph{Contribution 4: Learning and evaluation of trip recovery policies}
The last contribution is a method to learn and evaluation of trip response
policies for recovering from disturbances during swing. The learned policies
will advance the state-of-the-art as they will be the first trip recovery
policies implemented on real prosthesis hardware, whereas previous policies were
trained and tested offline using data collected from healthy human subjects.

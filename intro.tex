\chapter{Introduction}

\begin{marginfigure}
	\begin{subfigure}[b]{\textwidth}
    	\centering
        %\includegraphics{}
        \missingfigure{Pacifica}
        \caption{Freedom Innovations Pacifica ankle prosthesis}
        \label{fig:freedom_innov_pacifica}
        \vspace{0.25in}
	\end{subfigure}
	\begin{subfigure}[b]{\textwidth}
    	\centering
        %\includegraphics{}
        \missingfigure{C-Leg}
        \caption{Ottobock C-Leg knee prosthesis}
        \label{fig:ottobock_cleg}
	\end{subfigure}
    \caption{Examples of common mechanically-passive ankle and knee prostheses}
\end{marginfigure}

\Paragraph{Prevalance of Amputation}

There are currently an estimated six hundred thousand lower-limb amputees in the
United States, a number expected to rise significantly over the coming decades
due to the prevalence of diabetes~\citep{ziegler2008estimating}. Currently, prostheticists typically prescribe amputees mechanically-passive prostheses.  While mechanically passive knee prostheses can feature intelligence and control in the form of online adjustment of joint damping, they are incapable of providing positive joint work over the course of the gait cycle. The lack of active control of the knee and ankle joints results in three major consequences for amputee gait: increased energy consumption, unnatural gait kinematics, and increased likelihood of falling.

Popular options include the Freedom Innovations Pacifica~(\cref{fig:freedom_innov_pacifica}) and Ottobock C-Leg~(\cref{fig:ottobock_cleg}).
\section{Motivation}
\section{Challenges}
\section{Expected Contributions}
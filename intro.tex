\chapter{Introduction}
\begin{marginfigure}
    \centering
	\begin{subfigure}[b]{\textwidth}
    	\centering
        %\includegraphics{}
        \missingfigure{Pacifica}
        \caption{Freedom Innovations Pacifica ankle prosthesis}
        \label{fig:freedom_innov_pacifica}
        \vspace{0.25in}
	\end{subfigure}
	\begin{subfigure}[b]{\textwidth}
    	\centering
        %\includegraphics{}
        \missingfigure{C-Leg}
        \caption{Ottobock C-Leg knee prosthesis}
        \label{fig:ottobock_cleg}
	\end{subfigure}
    \caption{Examples of common mechanically-passive ankle and knee prostheses}
\end{marginfigure}

\begin{enumerate}
    \item Significance of problem
    \begin{enumerate}
        \item Number of amputees and cause of amputations
        \item Amputees face problems due to energy expenditure, unnatural gait,
            fear of falling gait
    \end{enumerate}
    \item Caused by mechanically passive prostheses. The tasks that we need to
    do: regular walking, upslope, downslope, stairs, tripping, stumbling require
    positive energy consumption

    \item People have tried to fix this with active prostheses over the past few
    years:
    \begin{enumerate}
        \item vanderbilt prostheses: slope walking, upstairs, other examples
        \item biom - upslope adaptation, biom dancing
    \end{enumerate}

    \item Control issue with impedance control.  Need many specialized
    controllers. Example different gains for slopes, rough ground, different
    target landing angles. Highlights need for more robust control approach.

    \item Discuss the constraints of prosthesis control:
    \begin{enumerate}
        \item decentralized - we do not wish to sensorize the whole body so
            traditional centralized robotics approaches are inapplicable
        \item dynamicism - human locomotion even walking is very dynamic, cop
        goes to edge of foot, straight knee. etc
        \item decentralized control approaches such as simbicon, impedance
        control, neuromuscular control - has already shown its robustness and
        ability to generalize and also produce natural gaits in simulation
    \end{enumerate}

    \item Approach - neuromuscular inspired prosthesis control
    \begin{enumerate}
        \item First test ideas using simulations of amputees before transferring
        to prosthesis. 
        \item Develop prosthesis hardware that can accurately reproduce the
        torques desired by the model
        \item Tune the parameters of the control for individual users using
        preferences and proposed dagger solution for trip recovery
    \end{enumerate}

    \item Expected contributions
    \begin{enumerate}
        \item SEA prosthesis design capable of producing enough torque for trip
        recovery, impact resilient, torque control.
        \item Evaluation of Neuromuscular prosthesis control in terms of
        energetics, kinematics, and preferences.
        \item method for tuning prostheses and other systems using preferences
        \item Dagger tuning method for learning high level trip recovery policy.
    \end{enumerate}
\end{enumerate}

\section{Motivation}
There are currently an estimated six hundred thousand lower-limb amputees in the
United States~\citep{ziegler2008estimating}. People suffer from amputations due
to traumatic injuries workplace accidents, traffic collisions, and as casualties
of war. However, a large percentage (54\%) suffer from the loss of a limb due to
complications arising from dysvacular disease associated with Diabetes.
Therefore, \citet{ziegler2008estimating} estimate that by 2050 the number of
amputees living in the United States will likely double, largely due to the
growing prevalence of diabetes in society.


Currently,
most amputees receive mechanically-passive prostheses, such as the Freedom
Innovations Pacifica~(\cref{fig:freedom_innov_pacifica}) ankle prosthesis or 
Ottobock C-Leg~(\cref{fig:ottobock_cleg}) knee prosthesis. While mechanically
passive knee prostheses can feature intelligence and control in the form of
online adjustment of joint damping, they are incapable of providing positive
joint work over the course of the gait cycle. Consequently, lower-limb amputees,
and especially transfemoral or above-the-knee amputees, equipped with
mechanically-passive prostheses suffer from a number of issues including
markedly increased energy consumption~\citep{}, unnatural gait
kinematics~\citep{}, and an increased likelihood of falling~\citep{}.
Importantly, these gait pathologies can lead an avoidance of walking, causing
a degradation of physical fitness and further deterioration of the amputee's
condition.

\begin{marginfigure}
    \centering
    %\includegraphics{}
    \missingfigure{Biom Ankle}
    \caption{Biom Robotic Ankle Prosthesis}
    \label{fig:biom_ankle}
\end{marginfigure}

To help remedy this situation, in the past decade researchers and companies have
developed robotic powered knee and ankle prostheses for lower-limb amputees.
These prostheses feature actuators at the knee and/or ankle that, if controlled
correctly, promise to restore the kinetics, kinematics, and reactions of the
healthy human leg. Noteworthy examples include four generations of transfemoral
prostheses developed by Vanderbilt University~\cref{fig:vanderbilt_prostheses}
~\citep{} and the Biom powered ankle~\cref{fig:biom_ankle}~\citep{}. These
powered prostheses have helped amputees \todo{cite improvements to amputee gait}

\begin{figure*}
    \centering
	\begin{subfigure}[b]{0.2\textwidth}
    	\centering
        %\includegraphics{}
        \missingfigure{Gen 1}
        \caption{Generation 1}
	\end{subfigure}
	\begin{subfigure}[b]{0.2\textwidth}
    	\centering
        %\includegraphics{}
        \missingfigure{Gen 2}
        \caption{Generation 2}
	\end{subfigure}
	\begin{subfigure}[b]{0.2\textwidth}
    	\centering
        %\includegraphics{}
        \missingfigure{Gen 3}
        \caption{Generation 3}
	\end{subfigure}
	\begin{subfigure}[b]{0.2\textwidth}
    	\centering
        %\includegraphics{}
        \missingfigure{Gen 4}
        \caption{Generation 4}
	\end{subfigure}
    \caption{Vanderbilt University's Robotic Transfemoral Prostheses}
\end{figure*}

However, it still remains an open research question how best to control these
prostheses to achieve natural and robust gaits. In the most established control
method for powered prostheses, the prosthesis uses simple impedance functions to
approximate the joint torque versus angle relationships observed during
walking~\citep{sup2009preliminary}. However, since the torque functions only
approximate steady, level walking, this method does not seem to generalize well
to other situations such as walking on slopes~\citep{sup2011upslope} or rough
ground~\citep{thatte2016toward} and changing foot placement
targets~\citep{schepelmann2016evaluation}. 

An alternative approach to joint control in prostheses is to mimic the
underlying dynamics and control of the human neuromuscular system. Rather than
replicating recorded torque profiles with impedance functions, modeling the
dynamical system that generates these torques may help generalize the control to
unexpected situations. Prior work on neuromuscular models shows that they can
lead to robust and natural-looking gaits when used to control a simulated biped.
For example, using a neuromuscular model, an optimized simulated biped model
walked on unseen, uneven terrain with sudden drops and steps up to 14
centimeters~\cite{song2015neural}. In addition, \citet{eilenberg2010control}
successfully applied the neuromuscular control approach to a powered ankle
prosthesis, which mimics the kinematics and kinetics of the ankle joint in human
walking including its adaptation to sloped environments. It remains unclear,
however, whether we can generalize the approach to the more complex function of
the knee joint in gait and balance recovery.

\subsection{Approach}
\section{Expected Contributions}

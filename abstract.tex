\chapter*{Abstract}

We present work towards developing a control method for powered knee and ankle
prostheses based on a neuromuscular model of human locomotion. Previous research
applying neuromuscular control to simulated biped models and to powered ankle
prostheses suggest that this approach can adapt to changes in speed, incline,
and rough ground. The improved robustness and generalizability of the approach
may arise from its modeling of various physical and neural components of the
human neuromuscular system. For example, research has shown that muscular reflexes, 
such as positive force feedback, can generate human-like compliant leg behavior, 
that muscle properties such as the force-velocity relationship are important for
regulating energy in simplified gait models, and that biarticular structures
play an important role in preventing joint over extension during compression of
multi-segmented legs. While research has demonstrated that these components
individually contribute to the robustness of simplified legged systems, it is
unclear if their combined effect when applied to prostheses will help improve
the amputee gait robustness. Therefore, the goal of this thesis is to
investigate how to apply neuromuscular control to a powered knee and ankle
prosthesis and quantify the robustness of amputee gait under this control
strategy.

To further motivate our use of neuromuscular control, we first model and
simulate an amputee walking with a powered prosthesis and perform optimizations
to obtain parameters for the proposed neuromuscular control and the established
impedance control method for prostheses. We find that neuromuscular control
significantly improves the simulated amputee's gait robustness on uneven ground.
To confirm that this improved robustness is evident on a real system, we design
and build a powered knee and ankle prosthesis that features powerful actuators
capable of producing sufficient torque and speed for trip recovery and series
elasticity to enable accurate reproduction of the neuromuscular model torques.
In parallel, we have investigated methods to optimize prosthesis control
parameters for specific subjects via qualitative feedback. In completed work, we
present and evaluate the performance of a Bayesian optimization method that
works with a user's preferences between pairs of parameters.

In our proposed work we intend to implement the neuromuscular control on the
completed prosthesis hardware and evaluate its robustness properties. We
hypothesize that the proposed control will allow amputees to more quickly
recover from disturbances. Furthermore, we will extend our method for optimizing
control parameters to include more forms of qualitative feedback and explicit
consideration of user adaptation over time. Finally, we propose to improve the
neuromuscular control's response to disturbances during swing via explicit
detection, classification, and execution of recovery strategies.

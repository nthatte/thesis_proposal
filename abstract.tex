\chapter*{Abstract}

We present work on developing a control method for powered knee-ankle prostheses
based on a neuromuscular model of human locomotion. Previous research applying
neuromuscular control to simulated biped models and to powered ankle prostheses
suggest that this approach may adapt automatically to changes in speed, incline,
and rough ground. The improved robustness and generalizability of the approach
may arise from its modeling of various physical and neural components of the
human neuromuscular system. For example, muscular reflexes, such as positive
force feedback, have been shown to generate human-like compliant leg behavior in
segmented legs. Research has shown muscle properties such as the force-velocity
relationship are important for regulating energy in simplified gait models.
Finally, biarticular structures have been shown to play an important role in
preventing joint over extension during compression of multi-segmented legs.
While research has demonstrated that these components individually contribute to
the robustness of simplified legged- systems, it is unclear if their combined
effect when applied to a powered prosthesis will help improve the amputee gait
robustness. Therefore, the goal of this thesis is to investigate how to apply
neuromuscular control to a powered knee and ankle prosthesis and quantify the
robustness of amputee gait under this control strategy.

To further motivate our use of neuromuscular control, we first model and
simulate an amputee walking with a powered prosthesis and compare the gait
robustness achieved by proposed neuromuscular control and the established
impedance control method. We find that the proposed control significantly
improves the simulated amputee's gait robustness on uneven ground. To confirm
that this improved robustness is evident on a real system, we design and build a
powered knee and ankle prosthesis. The prosthesis features on-board actuation to
allow eventual use outside of laboratory environments as well as series elastic
actuation to enable accurate torque control. With this prosthesis, we propose to
implement neuromuscular control and investigate its robustness properties. In
parallel, we have investigated methods to optimize prosthesis control parameters
for specific subjects via qualitative feedback. In completed work, we present 
and evaluate the performance of a Bayesian Optimization method that works with
a user's preferences between pairs of parameters.

In our proposed work we intend to implement the neuromuscular control on the
prosthesis hardware and evaluate its robustness properties. We hypothesize that
the proposed control will allow amputees to more quickly recover from
disturbances. Furthermore, we will extend our method for optimizing control
parameters to include more forms of qualitative feedback and explicit
consideration of user adaptation over time. Finally, we propose to improve the
neuromuscular control's response to disturbances during swing via explicit
detection, classification, and execution of recovery strategies.
